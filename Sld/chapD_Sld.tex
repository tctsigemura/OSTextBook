%\documentclass[dvipdfmx]{beamer}
\documentclass{beamer}                   % lualatex の場合
\usepackage{mySld}

\begin{document}
\title[主記憶]{オペレーティングシステム\\第13章 仮想記憶}
\date{}

\begin{frame}
  \titlepage
\end{frame}

%\section{}
%=========================================================================
%\begin{frame}
%  \frametitle{}
%\end{frame}

\section{仮想記憶}
%=========================================================================
\begin{frame}
  \frametitle{基本概念}
  ページングをベースに仮想記憶を実現する.
  \begin{itemize}
  \item システムの使用メモリ合計が物理メモリより大きい. → 実行可
  \item 単一のプログラムがメモリより大きい。 → 実行可
  \item ページテーブルのV=0を上手く使用する.
  \item V=0のページにアクセスするとページ不在割込み → OSへ
  \item プロセステーブルのV=0に二つの場合がある.
    \begin{itemize}
      \item[1.] 無効な領域 → プロセス終了
      \item[2.] \emph{バッキングストア}に退避中 → 復旧して再開
    \end{itemize}
  \item プロセス生成時にバッキングストアにプロセスのイメージを作る.
  \item Windows,macOS,Linux等,現代のOSのほとんどが採用している.
  \end{itemize}
\end{frame}

%=========================================================================
\begin{frame}
  \frametitle{仮想記憶の基本}
  \fig{scale=0.45}{virtualMemoryBasic-crop.pdf}
\end{frame}

%=========================================================================
\begin{frame}
  \frametitle{デマンドページング(Demand Paging)}
  \fig{scale=0.45}{virtualMemoryWithZMagic-crop.pdf}
  \begin{itemize}
  \item ページをswap-inするための方式の一つ.
  \item 全てのページが不在の状態からスタートする.
  \item ページ不在を起こしたページをswap-inする.\\
    (使用しないページを読み込むような無駄が無い)
  \end{itemize}
\end{frame}

%=========================================================================
\begin{frame}
  \frametitle{プログラムファイルの直接swap-inによる実行}
  \fig{scale=0.45}{virtualMemoryWithZMagic-crop.pdf}
  \begin{itemize}
  \item デマンドページング用の実行可能形式ファイルを用いる.\\
    (このファイルはページサイズを意識した構造になっている)
  \item プログラムはファイルからswap-inする(R-Xに設定).
  \item 初期化データはファイルからswap-inする(RW-に設定).
  \item 非初期化データ,ヒープ,スタックはゼロにする(RW-に設定).
  \end{itemize}
\end{frame}

%=========================================================================
\begin{frame}
  \frametitle{プログラムのswap-out}
  \fig{scale=0.45}{virtualMemoryWithZMagic-crop.pdf}
  \begin{itemize}
  \item フレームが枯渇したら使用頻度の低いフレームを解放し再利用する.
  \item プログラム(R-X)は変化しないのでswap-outしない.
  \item 初期化データ(RW-)はバッキングストアにswap-outする.
  \item 非初期化データ,ヒープ,スタックもswap-outする.
  \end{itemize}
\end{frame}

%=========================================================================
\begin{frame}
  \frametitle{Copy on Write(1)}
  \fig{scale=0.40}{virtualMemoryFork-crop.pdf}
  \begin{itemize}
  \item fork-exec ではアドレス空間のコピーに無駄が多い. → vfork
  \item vforkは使いにくい.使いやすいforkを改良する.
  \item forkの後,書き込み可能ページを一時的に\texttt{R--}に設定しておく.
  \end{itemize}
\end{frame}

%=========================================================================
\begin{frame}
  \frametitle{Copy on Write(2)}
  \fig{scale=0.40}{virtualMemoryCOW-crop.pdf}
  \begin{itemize}
  \item 例えばスタックに\emph{書き込み}があるとメモリ保護割込みが発生する.
  \item この時点でOSが新しいフレームを割当て,内容を\emph{コピー}する.
  \item ページを\texttt{RW-}に変更しプロセスを再開する.
  \end{itemize}
\end{frame}

%=========================================================================
\begin{frame}
  \frametitle{メモリマップドファイル(1)}
  \fig{scale=0.45}{virtualMemoryWithMmap-crop.pdf}
  \begin{itemize}
  \item 仮想記憶機構を用いたファイルへのアクセス手段である.
  \item プロセスはメモリ上の配列のようにファイルにアクセスできる.
  \item ファイルアクセスで,一々システムコールを使用しない.\\
    (軽いファイルアクセス手段)
  \item 同じファイルを複数プロセスがマッピング → 共有メモリになる.
  \end{itemize}
\end{frame}

%=========================================================================
\begin{frame}[fragile]
  \frametitle{メモリマップドファイル(2)}
  UNIX のメモリマップドファイルの例(mmapシステムコール)

\begin{lstlisting}[numbers=none]
void * mmap(void *addr, size_t len, int prot, int flags, int fd, off_t offset);
\end{lstlisting}

  \begin{description}
  \item[\emph{戻り値}:] マップされた領域の先頭アドレスが返される.
  \item[\texttt{addr}:] マップしたい仮想アドレス空間の先頭アドレスを渡す.
  \item[\texttt{len}:] マップする領域の大きさを渡す.
  \item[\texttt{prot}:] 保護モード(protection:RWX)を表す値を渡す.
  \item[\texttt{flags}:]共用する(\|MAP_SHARED|)/
    しない(\|MAP_PRIVATE|)等
  \item[\texttt{fd}:] オープン済みファイルのファイルディスクリプタを渡す.
  \item[\texttt{offset}:] ファイル中のマッピング位置.
  \end{description}

  アドレスや長さはページサイズの整数倍にする.
\end{frame}

%=========================================================================
\begin{frame}
  \frametitle{メモリマップドファイル(3)}
  \lst{numbers=none}{mmapTest.c}
\end{frame}

%=========================================================================
\begin{frame}
  \frametitle{メモリマップドファイル(4)}
  メモリマップドファイルの仕組み
  \fig{scale=0.45}{virtualMemoryWithMmap-crop.pdf}
  \begin{itemize}
  \item ファイルの読み込みはデマンドページングの要領で行う.
  \item ファイルの書き込みは
    \begin{itemize}
      \item Dirty ページを定期的にファイルに書き戻す.
      \item プロセスの終了やマッピングの解消時に書き戻す.
    \end{itemize}
  \end{itemize}
\end{frame}

%=========================================================================
\begin{frame}
  \frametitle{メモリマップドファイル(5)}
  read/writeシステムコールとの比較
  \fig{scale=0.45}{mmapVsReadWrite-crop.pdf}
  \begin{itemize}
  \item ファイルを操作する度にシステムコールを発行する.\\
    (システムコールは重い処理)
  \item ディスクキャッシュとプログラムのバッファ間でメモリコピー\\
    (メモリコピーは重い処理)
  \end{itemize}
\end{frame}

%=========================================================================
\begin{frame}[fragile]
  \frametitle{メモリマップドファイル(6)}
  プロセスにローカルなマッピング
  \fig{scale=0.45}{virtualMemoryWithPrivateMap-crop.pdf}
  \begin{itemize}
  \item これまでは\|MAP_SHARD|の例だった.
  \item \|MAP_PRIVATE|の例を紹介する.
  \item 最初は「ファイルデータ1」のように共有される(\texttt{R--}).
  \item 書き換えが発生した時点でコピーを作る(Copy on Write).
  \item 「ファイルデータ2」のようにプロセスは別々のコピーを参照する.
  \end{itemize}
\end{frame}

%=========================================================================
\begin{frame}[fragile]
  \frametitle{メモリマップドファイル(7)}
  プログラムの実行とメモリマップドファイル
  \fig{scale=0.40}{virtualMemoryWithZMagic-crop.pdf}
  \begin{itemize}
  \item 実行形式ファイルをメモリにマッピングする.
  \item プログラムは,\|R-X|,\|MAP_SHARD|でマッピングする.\\
    (プログラムはプロセス間で共用される)
  \item 初期化データは,\|RW-|,\|MAP_PRIVATE|でマッピングする.
  \item 非初期化データ,ヒープ,スタックはファイルにマッピングしない.
  \end{itemize}
\end{frame}

%=========================================================================
\begin{frame}
  \frametitle{練習問題}
  \begin{enumerate}
  \item[(1)] 次の言葉の意味を説明しなさい.
    \begin{itemize}
      \item 仮想記憶
      \item デマンドページング
      \item swap-in,swap-out
      \item Copy on Write
      \item メモリマップドファイル
    \end{itemize}
  \end{enumerate}
\end{frame}

\end{document}
