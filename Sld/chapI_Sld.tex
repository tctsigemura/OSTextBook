%\documentclass[dvipdfmx]{beamer}
\documentclass{beamer}                   % lualatex の場合
\usepackage{mySld}
\newcommand{\inode}{\texttt{i-node}}
\begin{document}
\title[ZFS]
      {オペレーティングシステム\\第18章 ZFS}
\date{}
\begin{frame}
  \titlepage
  \centerline{\url{https://github.com/tctsigemura/OSTextBook}}
\end{frame}

%\section{}
%=========================================================================
%\begin{frame}
%  \frametitle{}
%\end{frame}

\section{ZFS}
%=========================================================================
\begin{frame}
  \frametitle{ZFSの特徴(1)}
  2005年にサン・マイクロがOpenSolarisに実装して公開し,
  オープンソースで開発が続いているファイルシステム.
  FreeBSD,Linux等に移植されSolaris以外のOSでも使用できるようになっている.
  大きな主記憶と,高速なマルチプロセッサシステムを前提に設計されている.

  \begin{itemize}
  \item \emph{COW(Copy On Write)}でデータやメタデータを
    ハードディスク(以下ではデバイス)に書き込む.
    デバイスのブロックを上書きすることが無い.
  \item 一連の書き込み終了時点にUberblockを書き込むと変更が反映される.
  \item Uberblockの書き込み前なら変更前の完全な状態,
    Uberblockの書き込み後なら変更後の完全な状態になり,
    変更途中の不完全な状態になることはない.
  \item チェックサムにより高い信頼性が確保されている.
    ファイルシステムのメタデータだけでなく,
    全てのデータ(ブロック)のチェックサムが,
    そのブロックを管理する1階層上のデータ構造に記録されている.(次ページの図)
  \end{itemize}
  \vfill
\end{frame}

%=========================================================================
\begin{frame}
  \frametitle{\inode 全ブロックにわたるチェックサム}
  \fig{scale=0.5}{zfsCheckSum-crop.pdf}
  \begin{itemize}
    \item ブロックポインタがチェックサムを持つ.
    \item チェックサムの不整合が見つかった場合,
      データの2重化(ミラー)がされていれば,
      自動的にミラーからデータを修復する.
  \end{itemize}
  \vfill
\end{frame}

%=========================================================================
\begin{frame}
  \frametitle{ZFSの特徴(2)}
  \begin{itemize}
  \item \emph{スナップショット}や\emph{クローン}の作成が一瞬で完了する.
    その後は COW の手法を使用し,
    コピーとオリジナルに違いが出た時点で,
    違いが出たブロックとそれの親だけのコピーが作られる.
    デバイスの容量も無駄にならない.(前の図で最上位だけコピーするイメージ)
  \item ボリュームの代わりにストレージプールと呼ばれる
    ソフトウェアの層をデバイスとファイルシステムの間にはさんでいる.
    (次々ページ)
  \item ファイルサイズ等の制約が事実上無くなった.
    ファイルサイズは最大$2^{64}$バイト,
    ストレージプールサイズは最大$2^{70}$バイト($Zetta = 2^{70}$ )
  \item ストレージプールは,
    ミラーやRAID-Z等によりデバイスの故障に対する信頼性・可用性を向上する.
  \end{itemize}
  \vfill
\end{frame}

%=========================================================================
\begin{frame}[fragile]
  \frametitle{従来の方式}
  \fig{scale=0.6}{zfsVolume-crop.pdf}
  \begin{itemize}
  \item ファイルシステムの初期化以前にボリュームを決定し,
  \item 後でサイズの変更などはできない.
  \end{itemize}
  \vfill
\end{frame}
  
%=========================================================================
\begin{frame}[fragile]
  \frametitle{ストレージプール}
  \fig{scale=0.6}{zfsPool-crop.pdf}
  \begin{itemize}
  \item ストレージプールは沢山のデバイスを収容する.
  \item ZFSからの要求に応じてデータブロックを割り付ける.
  \item C言語プログラムの\|malloc()|や\|free()|に似ている.
  \item ストレージプールに後でデバイスを追加することも可能できる.
  \end{itemize}
  \vfill
\end{frame}
  
%=========================================================================
\begin{frame}
  \frametitle{ZFSの特徴(3)}
  \begin{itemize}
  \item 仮想記憶のページキャッシュと統合されていない.
  \item CPUやメモリの利用率が高い.
    64ビットCPUでないとZFSに十分なメモリを提供できない.
    (FreeNASでは最低8GiBのメモリ)
  \end{itemize}
  \vfill
\end{frame}

%=========================================================================
\begin{frame}[fragile]
  \frametitle{ZFSのソフトウェア構成(1)}
  \fig{scale=0.45}{zfsSoftModule-crop.pdf}
  \begin{enumerate}
  \item[1.] システムコールは,OSカーネル本体がVNODE操作に変換する.
  \item[2.] ZPLは VNODE 操作をZFSのトランザクションに変換する.
    1つのシステムコールが1つのトランザクションに変換される.
  \end{enumerate}  
  \vfill
\end{frame}

%=========================================================================
\begin{frame}[fragile]
  \frametitle{ZFSのソフトウェア構成(2)}
  \fig{scale=0.45}{zfsSoftModule-crop.pdf}
  \begin{enumerate}
  \item[3.] DMUは複数トランザクションを
    \emph{トランザショングループ}にする.
  \item[4.] SPAは,DMUがトランザクショングループをキャッシュに書込み終わると,
    キャッシュの内容をデバイスに反映させる.
  \end{enumerate}  
  \vfill
\end{frame}

%=========================================================================
\begin{frame}[fragile]
  \frametitle{ZFSのソフトウェア構成(3)}
  \fig{scale=0.45}{zfsTranscation-crop.pdf}
  \begin{enumerate}
  \item[3.] DMUは複数トランザクションを
    \emph{トランザショングループ}にする.
  \item[4.] SPAは,DMUがトランザクショングループをキャッシュに書込み終わると,
    キャッシュの内容をデバイスに反映させる.(バースト)
  \end{enumerate}  
  \vfill
\end{frame}

%=========================================================================
\begin{frame}[fragile]
  \frametitle{ストレージプールの構造(概要)}
  \centerline{
  \begin{minipage}{0.45\columnwidth}
    デバイス内部の構造
    \fig{scale=0.6}{zfsDevice.pdf}
  \end{minipage}
  \begin{minipage}{0.45\columnwidth}
    ボリュームラベル
    \fig{scale=0.6}{zfsVolumeLabel.pdf}
  \end{minipage}
  }
  \begin{itemize}
  \item デバイス(ディスク)の4箇所に同じボリュームラベルを書く.
  \item ボリュームラベルには128個のUberblockを格納できる.
  \item Uberblockはトランザクショングループ番号を含んでいる.
  \item Uberblockはトランザクショングループ番号を128で割った余りの位置に書く.
  \end{itemize}
  \vfill
\end{frame}

%=========================================================================
\begin{frame}[fragile]
  \frametitle{ストレージプールの更新(1)}
  \fig{scale=0.45,clip,trim=0 290 0 0}{zfsCommit-crop.pdf}
  \begin{enumerate}
    \item[1.] Uberblock起点の木構造でブロックは記録されている.
    \item[2.] 変更するには,新しいブロックを確保し内容を書き込む(COW).
  \end{enumerate}
  \vfill
\end{frame}

%=========================================================================
\begin{frame}[fragile]
  \frametitle{ストレージプールの更新(2)}
  \fig{scale=0.45,clip,trim=0 0 0 288}{zfsCommit-crop.pdf}
  \begin{enumerate}
    \item[3.] メタデータブロックもCOWで更新する.
    \item[4.] Uberblockを新しい領域に書き込む.\\
      (トランザクショングループ番号が最新のUberblockが有効)\\
      (古い世代のブロックは解放され,再利用される.)
  \end{enumerate}
  \vfill
\end{frame}

%=========================================================================
\begin{frame}[fragile]
  \frametitle{ブロックポインタ}
  図中でブロックを指していた\emph{矢印}を表現するデータ構造を
  \emph{ブロックポインタ}と呼ぶ.
  ブロックポインタはデータ多重化のために最大3組のアドレスを記録できる.
  ブロックポインタの内容は以下の通り.
  \vfill
  \begin{itemize}
  \item \emph{サイズ}:ブロックの大きさに関する情報
  \item \emph{チェックサム}(64ビット):ブロックのチェックサム(最大3個)
  \item \emph{ブロックのアドレス}:
    ブロックのストレージプール内での格納位置に関する情報(最大3個)
    (デバイス,デバイス内アドレス)
  \item \emph{タイムスタンプ}:
    ブロックを作成したトランザクショングループの番号
    (ブロックが削除される時にスナップショットと比較)
  \item \emph{その他}:チェックサム計算に使用するアルゴリズムの種類,
    データ圧縮に使用するアルゴリズムの種類,圧縮後のサイズなど...
  \end{itemize}
  \vfill
\end{frame}

%=========================================================================
\begin{frame}[fragile]
  \frametitle{Dnode(1)}
  ストレージプール内のあらゆるオブジェクトを表現する
  512バイトのデータ構造である.
  USFの\texttt{i-node}に似ているが,
  ファイルやディレクトリだけでなく,ファイルシステムなども表現する.
  \fig{scale=0.55}{zfsDnode-crop.pdf}
  \vfill
\end{frame}

%=========================================================================
\begin{frame}[fragile]
  \frametitle{Dnode(2)}
  \fig{scale=0.55}{zfsDnode-crop.pdf}
  \begin{itemize}
  \item dnodeは三つ以内のブロックポインタを格納することができる.
  \item dnodeは表現するオブジェクトに応じたデータを格納する領域を持っている.
    (この領域はブロックポインタと共用になっている)
  \end{itemize}
  \vfill
\end{frame}

%=========================================================================
\begin{frame}[fragile]
  \frametitle{Dnode(3)}
  \fig{scale=0.55}{zfsDnode-crop.pdf}
  \begin{itemize}
  \item データの大きさが128KiB以内の場合は直接参照(図の(a))
  \item 大きさが128KiBを超える場合は間接ブロック(図の(b))
  \end{itemize}
  \vfill
\end{frame}

%=========================================================================
\begin{frame}[fragile]
  \frametitle{Dnode(4)}
  \fig{scale=0.55}{zfsDnode-crop.pdf}
  \begin{itemize}
  \item 128KiBの間接ブロックはブロックポインタを最大1Ki個格納できる.
  \item $128KiB \times 1Ki = 128MiB$より大きなデーを表現する時は,
    多重の間接ブロック(最大6レベル)を用いる.($2^{64}$バイト以上)
  \end{itemize}
  \vfill
\end{frame}

%=========================================================================
\begin{frame}[fragile]
  \frametitle{ストレージプールの全体像}
  \fig{scale=0.4}{zfsStructure-crop.pdf}
  \vfill
\end{frame}

%=========================================================================
\begin{frame}[fragile]
  \frametitle{ストレージプールの全体像(MOS layer)}
  \fig{scale=0.4,clip,trim=0 250 0 0}{zfsStructure-crop.pdf}
  \begin{itemize}
  \item Objsetはdnodeの配列を管理する.
  \item 配列には,ストレージプールプール全体の管理に関する情報や,
    ファイルシステムやスナップショット等の一覧が格納される.
  \item master node はストレージプールのコンフィグ,プロパティなど.
  \item space map はストレージプール内のブロックの割付を管理する.
  \end{itemize}
  \vfill
\end{frame}

%=========================================================================
\begin{frame}[fragile]
  \frametitle{ストレージプールの全体像(Object-set layer)}
  \fig{scale=0.4,clip,trim=0 0 0 250}{zfsStructure-crop.pdf}
  \begin{itemize}
  \item ファイルシステムはObjsetのdnode配列で表現する.
  \item dnodeリストがUFSの\texttt{i-node}リストに相当する.
  \item master node は root ディレクトリの dnode 番号などの情報.
  \item 図は,通常ファイルの例とディレクトリファイルの例
  \end{itemize}
  \vfill
\end{frame}

%=========================================================================
\begin{frame}[fragile]
  \frametitle{スナップショットの作成}
  \fig{scale=0.4,clip,trim=0 0 0 200}{zfsSnapshot-crop.pdf}
  \begin{itemize}
  \item ファイルシステムを表すObjsetのコピーを作る.
  \item MOS layer の dnode 配列に登録する.
  \item スナップショットは一瞬で作成できる.
  \item 変更がないデータブロックは共用する.
  \item 内容が書き換わるとCOWで必要最小限のブロックのコピーを作る.
  \end{itemize}
  \vfill
\end{frame}

%=========================================================================
%\begin{frame}[fragile]
%  \frametitle{}
%  \vfill
%\end{frame}

\end{document}
