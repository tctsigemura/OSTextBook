%\documentclass[dvipdfmx]{beamer}
\documentclass[unicode,handout]{beamer}                   % lualatex の場合
\usepackage{mySld}

\begin{document}
\title[ページング]
      {オペレーティングシステム\\第11章 ページング}
\date{}
\begin{frame}
  \titlepage
  \centerline{\url{https://github.com/tctsigemura/OSTextBook}}
\end{frame}

%\section{}
%=========================================================================
%\begin{frame}
%  \frametitle{}
%\end{frame}

\section{ページング}
%=========================================================================
\begin{frame}
  \frametitle{ページング}
  ページングは以下のようなメモリ管理方式である.
  \begin{itemize}
  \item メモリを一様なページに分割し,ページ単位で管理する.
  \item メモリより広い仮想アドレス空間を使用できる.
  \item 外部フラグメンテーションを生じない.
  \item メモリコンパクションが不要である.
  \item Windows,macOS,Linux等,現代の多くのOSが採用している.
  \item 用語
    \begin{itemize}
      \item \emph{ページ}:仮想アドレス空間をページに分割する.
      \item \emph{フレーム}:物理アドレス空間をフレームに分割する.
    \end{itemize}
  \end{itemize}
\end{frame}

%=========================================================================
\begin{frame}
  \frametitle{基本概念}
  \fig{scale=0.4}{pageToFrame-crop.pdf}
  \begin{itemize}
  \item ページをフレームにマッピングする.
  \item ページサイズとフレームサイズは同じ.
  \item どのフレームも任意プロセスの任意ページにマッピングできる.
  \end{itemize}
\end{frame}

%=========================================================================
\begin{frame}
  \frametitle{ページとフレーム}
  \fig{scale=0.5}{pagingAddress-crop.pdf}
  \begin{itemize}
  \item 仮想アドレスの上位ビットがページ番号(p)
  \item 仮想アドレスの下位ビットがページ内アドレス(w)
  \item ページサイズは2の累乗にする.
  \item ページ内アドレスがwビットならページサイズは$2^w$バイト
  \item 物理アドレスの上位ビットがフレーム番号(f)
  \item 物理アドレスの下位ビットがフレーム内アドレス(w)
  \item ページ内アドレスとフレーム内アドレスは同じ(w)
  \item p を f に変換することでページをフレームにマッピングする.
  \item p と f のビット数は異なっていても良い.
  \end{itemize}
\end{frame}

%=========================================================================
\begin{frame}
  \frametitle{マッピング関数}
  \fig{scale=0.35}{pageToFrame-crop.pdf}
  \begin{itemize}
  \item p → f 変換関数をマッピング関数と呼ぶ.
  \item \emph{ページテーブル}(表)を用いて実装する.
  \item MMUがページテーブルを参照してマッピングする.
  \item プロセス毎にマッピング関数は異なる.
  \item ディスパッチャがMMUを操作しマッピングを入れ換える.
  \end{itemize}
\end{frame}

%=========================================================================
\begin{frame}
  \frametitle{フラグメンテーション}
  \begin{minipage}{0.35\columnwidth}
    \fig{scale=0.5}{pagingInnerFragment-crop.pdf}
  \end{minipage}
  \begin{minipage}{0.63\columnwidth}
    \begin{itemize}
    \item \emph{外部フラグメンテーション}は解決した.
      \begin{itemize}
      \item どのフレームでも任意のプロセスの任意のページにマッピングできる.
      \item メモリコンパクションも不要になった.
      \end{itemize}
    \item \emph{内部フラグメンテーション}が発生する.\\
      ページ毎にメモリ保護モードを設定する.
      \begin{itemize}
        \item プログラム領域は\texttt{r-x}にする.
        \item データ領域は\texttt{rw-}にする.
        \item \emph{あな}部分にはフレームを割当てない.\\
          (sparse address spaces)
        \item スタック領域も\texttt{rw-}にする.
      \end{itemize}
      領域サイズはページの倍数ではない.\\
      フラグメント部分のアクセスは不正だが検知できない.\\
      フラグメントサイズの平均はページサイズの1/2.
    \end{itemize}
  \end{minipage}
\end{frame}

%=========================================================================
\begin{frame}
  \frametitle{ページング機構の概要}
  \fig{scale=0.4}{paging-crop.pdf}
  \begin{itemize}
  \item ページテーブルの一つの\emph{エントリ}をページ番号(p)で選択する.
  \item 選択したエントリに格納されているフレーム番号(f)を取り出す.
  \item フレーム番号(f)とページ内アドレス(w)を結合し物理アドレスにする.
  \end{itemize}
\end{frame}

%=========================================================================
\begin{frame}
  \frametitle{ページテーブルエントリ}
  \begin{itemize}
  \item \emph{fフィールド}:フレーム番号
  \item \emph{cフィールド}:制御 \\
    \tbl{scale=0.9}{pageTableAttr.pdf}
    \begin{itemize}
    \item V=0 なら\emph{ページ不在割込み}
    \item R はページの使用頻度の測定等に使用
    \item D=0 ならスワップアウト不要
    \item RWX によりメモリ保護(\emph{メモリ保護割込み})
    \end{itemize}
  \end{itemize}
\end{frame}

%=========================================================================
\begin{frame}
  \frametitle{ページテーブル}
  \fig{scale=0.35}{paging-crop.pdf}
  \begin{itemize}
  \item ページテーブルはかなり大きな表である.
  \item ページ番号が20ビットなら$2^{20}=1Mi$エントリ
  \item エントリのサイズが4バイトと仮定すると全体で4MiB
  \item プロセス毎に必要なのでディスパッチの度にロードするのも大変
  \item \emph{ページテーブルレジスタ}にアドレスを記録しメモリ上に置く
  \end{itemize}
\end{frame}

%=========================================================================
\begin{frame}
  \frametitle{TLB(Translation Look-aside Buffer)}
  \fig{scale=0.35}{pagingWithTlb-crop.pdf}
  \begin{itemize}
  \item メモリ上にあるページテーブルにアクセスするには時間がかかる.
  \item 変換結果(pとfの対応)を\emph{TLB}にキャッシュする.
  \item TLB(数十から数千エントリ)は高速な連想メモリ.
  \end{itemize}
\end{frame}

%=========================================================================
\begin{frame}
  \frametitle{TLB(Translation Look-aside Buffer)}
  \fig{scale=0.35}{pagingWithTlb-crop.pdf}
  \begin{enumerate}
  \item[1)] ページ番号(p)でTLBを検索しエントリを選択する.
    (\emph{TLB miss})
  \item[2)] RWXをチェックする.(メモリ保護例外)
  \item[3)] フレーム番号(f)を出力する.
  \end{enumerate}
\end{frame}

%=========================================================================
\begin{frame}
  \frametitle{Page Table Walk}
  \begin{itemize}
  \item \emph{TLB miss}のときページテーブルを検索すること.
  \item ハードウェアで自動的に行う場合 \\
    ページテーブルレジスタからページテーブルの位置を知る. \\
    ハードウェアを用いることで高速化
  \item ソフトウェアで行う場合 \\
    TLB miss 割込みを発生しOSに切換える. \\
    ハードウェアが単純 => チップ面積に余裕 => \\
    TLBのエントリ数を増やす => TLB miss の頻度を低くする.
  \item ページテーブルのエントリ V=0 の場合 \\
    ページ不在割込みを発生 \\
  \item ページテーブルのエントリ V=1 の場合 \\
    1. TLB のエントリの一つをページテーブルに書き戻す. \\
    2. TLB の空いたエントリにページテーブルからロード
  \end{itemize}
\end{frame}

%=========================================================================
\begin{frame}
  \frametitle{TLBエントリのクリア}
  \begin{itemize}
  \item プロセススイッチのとき
  \item ページテーブルに変更があったとき
  \item TLBの内容は変更前のページテーブルのエントリなので \\
    クリアする必要がある.
  \item TLBのクリアは\emph{大きなペナルティ}を伴うので避けたい.
  \item TLBのエントリがプロセス番号を含む方式
  \item TLBのエントリを個別にクリアできる方式
  \end{itemize}
\end{frame}

%=========================================================================
\begin{frame}
  \frametitle{フレームの共用}
  \fig{scale=0.4}{pageSharing-crop.pdf}
  \begin{itemize}
  \item プロセスが変更しないページ(\texttt{R-X})は共用できる.
  \item ページテーブルの操作により実現
  \item ライブラリは\emph{位置独立コード}でなければならい.
  \end{itemize}
\end{frame}

%=========================================================================
\begin{frame}
  \frametitle{位置独立コード}
  \begin{itembox}[l]{位置独立コードのイメージ}
    \begin{tabular}{l l l}
      CALL  & 200,PC    &   // 200番地先にあるサブルーチン実行    \\
      JMP   & 100,PC    &   // 100番地先にジャンプする            \\
      LD    & G0,4,FP   &   // ローカル変数はスタック上           \\
      ST    & G0,40,G11 &   // グローバル変数はレジスタ相対で参照 \\
    \end{tabular}
  \end{itembox}
  \begin{itemize}
  \item ライブラリはマッピングされる仮想アドレスが変化する.
  \item どのアドレスにマッピングされても大丈夫なプログラム \\
    => \emph{位置独立コード}
  \item PC相対でJMPやCALLを行う.
  \item データはレジスタをベースにアクセスする.
  \end{itemize}
\end{frame}

%=========================================================================
\begin{frame}
  \frametitle{ページテーブルの編成方法}
  \begin{itemize}
  \item ページテーブルは大きくなりすぎる.(32ビットCPUの例)
    \begin{itemize}
      \item 仮想アドレス32ビット,ページサイズ4KiB,エントリ4Bの例 \\
        $2^{32} \div 4KiB = 2^{32} \div 2^{12} = 2^{20} = 1Mi$エントリ\\
        $1Mi エントリ \times 4B = 4MiB$
      \item 32ビットCPUの普及が始まった当時のPCは, \\
        メモリを 4MiB 〜 16MiB しか搭載していなかった.
    \end{itemize}
    \vfill
  \item ページテーブルは大きくなりすぎる.(64ビットCPUの例)
    \begin{itemize}
      \item 仮想アドレス48ビット,ページサイズ4KiB,エントリ8Bの例 \\
        $2^{48} \div 4KiB = 2^{48} \div 2^{12} = 2^{36} = 64Gi$エントリ\\
        $64Gi エントリ \times 8B = 512GiB$
      \item 現代の64ビットPCのメモリは,4GiB 〜 16GiB 程度?
    \end{itemize}
    \vfill
  \item ページテーブルを小さくする工夫が必要!!
  \end{itemize}
\end{frame}

%=========================================================================
\begin{frame}
  \frametitle{二段のページテーブル(IA-32の例)}
  \fig{scale=0.33}{pagingMultiLevel-crop.pdf}
  \begin{itemize}
  \item 一段目のページテーブルサイズ4KiB = フレームサイズ
  \item 二段目のページテーブルの区画サイズ4KiB = フレームサイズ
  \end{itemize}
\end{frame}

%=========================================================================
\begin{frame}
  \frametitle{ページテーブルフレームの節約}
  \fig{scale=0.5}{pagingSparseSpace-crop.pdf}
  \begin{itemize}
  \item 仮想アドレス空間の\emph{あな}部分の二段目を省略
  \item 一段目のページテーブルエントリのV=0にする.
  \item 従来1,025フレーム => 3フレーム
  \end{itemize}
\end{frame}

%=========================================================================
\begin{frame}
  \frametitle{64ビット仮想アドレス空間(x86-64の例)}
  \begin{itemize}
  \item 実質48ビット仮想アドレス => 256TiB(十分大きい)
  \item 仮に二段のページテーブルならページテーブルの区画は \\
    18ビット(p),18ビット(q),12ビット(w) と仮定 \\
    エントリサイズ = 8B  と仮定 \\
    $2^{18} \times 8B = 2MiB$
  \item プロセスあたり最低でも3区画必要 \\
    $2MiB \times 3 = 6MiB$
  \item 400個のプロセスがあったとすると \\
    $6MiB \times 400 = 2.4GiB$(8GiBの30\%)
  \item 二段のページテーブルでは区画が大きくなりすぎる.
  \item 区画を小さくするために段数を多くする.
  \end{itemize}
\end{frame}

%=========================================================================
\begin{frame}
  \frametitle{四段のページテーブル(x86-64の例)}
  \fig{scale=0.4}{paging4Level-crop.pdf}
  \begin{itemize}
  \item ページサイズ(フレームサイズ)は4KiB
  \item ページテーブルの区画は$2^9 \times 8B = 4KiB$
  \item ページテーブルは最低7フレーム(28KiB)
  \item 400プロセスでも約11MiBで済む(8GiBの0.13\%)
  \end{itemize}
\end{frame}

%=========================================================================
\begin{frame}
  \frametitle{逆引きページテーブル}
  \fig{scale=0.4}{pagingInvertedTable-crop.pdf}
  \begin{itemize}
  \item テーブルでフレーム番号とページ番号の立場が逆転(逆引き)
  \item システム全体でページテーブル一つ(プロセス毎ではない)
  \item どの仮想アドレス空間のエントリか識別するための pid あり
  \end{itemize}
\end{frame}

%=========================================================================
\begin{frame}
  \frametitle{逆引きページテーブル}
  \emph{ページテーブルのサイズ}
  \begin{itemize}
  \item 8GiBのメモリを4KiBのページで分割する場合のエントリ数\\
    \centerline{$8GiB \div 4KiB = 2^{33} \div 2^{12} = 2Mi$エントリ}
  \item 1エントリ8バイト仮定すると
    \centerline{$2Mi \times 8B = 12MiB$}
  \item システム全体で12MiBで済む(8GiBの0.2\%)
  \end{itemize}
  \vfill
  \emph{Page Table Walk}
  \begin{itemize}
  \item CPUは仮想アドレスの他にpid(プロセス番号を出力)
  \item ページテーブルを pid と p(ページ番号)で探索する
  \item 線形探索などを用いると遅くて実用にならない
  \end{itemize}
\end{frame}

%=========================================================================
\begin{frame}
  \frametitle{逆引きページテーブル(IBM 801 の例)}
  \fig{scale=0.4}{paging801-crop.pdf}
  \begin{itemize}
  \item ハッシュ表を用いて探索を高速化
  \item ハッシュ表の大きさは二の累乗($0 〜 2^n-1$)
  \item ページテーブルはnext を使用してチェインを作る
  \end{itemize}
\end{frame}

%=========================================================================
\begin{frame}
  \frametitle{逆引きページテーブル(IBM 801 の例)}
  \emph{ハッシュ表を用いた Page Table Walk}
  \begin{itemize}
  \item pid と p を用いてハッシュ関数を計算する($h \leftarrow f(pid, p)$)\\
    (ハッシュ関数は pid と p の XOR。。。速度優先)
    \vfill
  \item ハッシュ表の第 h エントリを見る
    \begin{itemize}
      \item 空(図では×)ならページ不在(\emph{割込み!})
      \item 空でなければページテーブルのインデクス(f)
    \end{itemize}
    \vfill
  \item ページテーブルの第 f エントリの内容を見る
    \begin{itemize}
      \item pid ,p が一致 →  この時の f をフレーム番号にする(\emph{完了!})
      \item pid ,p が一致しない →  next を見る
        \begin{itemize}
        \item 空(図では×)ならページ不在(\emph{割込み!})
        \item 空でなければ\\ 
          ページテーブルのインデクス(f)を更新して再度トライ
        \end{itemize}
    \end{itemize}
  \end{itemize}

  \emph{TLB} : 不可欠!
\end{frame}

%=========================================================================
\begin{frame}
  \frametitle{練習問題}
  \begin{enumerate}
  \item[(1)] 一回のメモリアクセス時間に1ns,page table walk に2nsかかるとする,
    TLBのヒット率が90パーセントの時の平均メモリアクセス時間を計算しなさい.
  \item[(2)] 図11.7 において,
    $p=1$の仮想アドレスの範囲を8桁の16進数で答えなさい.
  \item[(3)] 図11.7 において,
    $p=1$,$q=1$の仮想アドレスの範囲を8桁の16進数で答えなさい.
  \item[(4)] 逆引きページテーブルを用いる場合,
    TLBに格納すべき最低限の情報の範囲を考察しなさい.
  \item[(5)] 図11.11 に,
    $pid=3$,$p=2$のページがフレーム1にマッピングされるような
    ページテーブルの状態を書き込みなさい.
  \item[(6)] 逆引きページテーブルを用いるシステムで,
    プロセス間でページの共有が可能か考察しなさい.
  \end{enumerate}
\end{frame}

\end{document}
