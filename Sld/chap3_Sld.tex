%\documentclass[dvipdfmx]{beamer}
\documentclass[unicode,handout]{beamer}                   % lualatex の場合
\usepackage{mySld}

\begin{document}
\title[CPUの仮想化]{オペレーティングシステム\\第3章 CPUの仮想化}
\date{}
\begin{frame}
  \titlepage
  \centerline{\url{https://github.com/tctsigemura/OSTextBook}}
\end{frame}

%=========================================================================
%\begin{frame}
%  \frametitle
%  \tableofcontents
%\end{frame}

%=========================================================================
\section{時分割多重}
\begin{frame}
  \frametitle{時分割多重によるCPUの仮想化}
  \fig{scale=0.7}{virtualCPU-crop.pdf}
  \begin{itemize}
    \item 時分割多重:CPUが実行するプロセスを次々切換える。
    \item コンテキストスイッチ:CPUが実行するプロセスを切換えること。
    \item ディスパッチ:プロセスにCPUを割り付ける。(実行開始)
    \item ディスパッチャ:ディスパッチするプログラムのこと。
  \end{itemize}
\end{frame}

%=========================================================================
\begin{frame}
  \frametitle{CPUの構造(参考)}
  \fig{scale=0.6}{cpuBlock-crop.pdf}
  \begin{itemize}
    \item コンテキスト = PSW + CPUレジスタ
    \item コンテキストを保存・ロードして次のプロセスに
    \item コンテキストスイッチ
  \end{itemize}
\end{frame}

%=========================================================================
\begin{frame}
  \frametitle{プロセスの構造(参考)}
  \fig{scale=0.5}{procOrganization-crop.pdf}
  \begin{itemize}
    \item 仮想CPUにコンテキストを保存
  \end{itemize}
\end{frame}

%=========================================================================
\section{プロセスの状態}
\begin{frame}
  \frametitle{プロセスの状態遷移}
  \fig{scale=0.6}{procState-crop.pdf}
  \begin{itemize}
    \item 基本的な三つの状態
    \item 六つの状態遷移
  \end{itemize}
\end{frame}

%=========================================================================
\section{プロセスの切換え(コンテキストスイッチ)}
\begin{frame}
  \frametitle{プロセス切換えの原因}
  \begin{itemize}
    \item イベント\\
      プロセス自ら「システムコールを発行する」Blockする \\
      他のプロセスから「\emph{干渉}を受け」Blockする    \\
      他のプロセスから「\emph{干渉}を受け」Preemptionする \\
      他のプロセスから「\emph{干渉}を受け」Completeする \\
      I/O完了やタイマ完了のイベントによりCompleteする
      \vfill
    \item タイムスライシング\\
      \emph{クオンタムタイム}を使い切ったプロセスはPreemptionする
  \end{itemize}
  \vfill
\end{frame}

%=========================================================================
\begin{frame}
  \frametitle{プロセスの切換え手順}
  \fig{scale=0.38}{procSwitch-crop.pdf}
\end{frame}

%=========================================================================
\begin{frame}
  \frametitle{オペレーティングシステムの構造(参考)}
  \fig{scale=0.4}{osOrganization-crop.pdf}
  \begin{itemize}
    \item 割込みハンドラ
    \item サービスモジュール
    \item ディスパッチャ
  \end{itemize}
\end{frame}

%=========================================================================
\begin{frame}
  \frametitle{プロセスの切換えの例}
  \fig{scale=0.34}{procSwitchInst-crop.pdf}
\end{frame}

%=========================================================================
\section{PCB(Process Control Block)}
\begin{frame}
  \frametitle{PCB(Process Control Block)}
  \begin{itemize}
    \item プロセスを表現するカーネル内データ構造
    \item プロセス毎に存在する
    \item カーネル内のプロセステーブルに格納される
  \end{itemize}
  \fig{scale=0.35}{procOrganization-crop.pdf}
  \vfill
\end{frame}

%=========================================================================
\begin{frame}
  \frametitle{PCBの内容}
\begin{itemize}
\item 仮想CPU
\item プロセス番号
\item 状態(Running,Waiting,Ready等)
\item 優先度
\item 統計情報(CPU利用時間等)
\item 次回のアラーム時刻
\item 親プロセス
\item 子プロセス一覧
\item シグナルハンドリング
\item 使用中のメモリ
\item オープン中のファイル
\item カレントディレクトリ
\item プロセス所有者のユーザ番号
\item PCBのリストを作るためのポインタ
\end{itemize}
\end{frame}

%=========================================================================
\begin{frame}
  \frametitle{PCBのリスト}
  \fig{scale=0.4}{procQueue-crop.pdf}
  \begin{itemize}
    \item Ready状態PCBのリスト = 実行可能列(優先順位順にソート)
    \item イベント毎のWaiting状態PCBのリスト = イベント待ち行列
  \end{itemize}
  \vfill
\end{frame}

%=========================================================================
\section{スレッド(Thread)}
\begin{frame}
  \frametitle{スレッド(Thread)}
  \begin{itemize}
    \item CPUの仮想化によりマルチプログラミングが可能になった.
    \item プロセスが並行(Concurrent)に実行できる.
    \item プロセスは\emph{一つ}の仮想CPUを持っている.
    \item プロセスはコンピュータを仮想化したもの.
      \begin{itemize}
      \item CPUが一つしかないコンピュータを仮想化している.
      \item CPUを複数持つSMPを仮想化するには不十分.
      \end{itemize}
    \item 一つのプロセスが複数の仮想CPUをもつモデルを導入する.
    \item プロセスが処理の流れ\emph{スレッド}を複数持つことができる.
  \end{itemize}
  \vfill
\end{frame}

%=========================================================================
\begin{frame}
  \frametitle{スレッドの役割(1)}
  マルチプログラミングを用いないWebサーバ \\
  \vfill
  \fig{scale=0.6}{singleProcSingleClient-crop.pdf}
  \vfill
  \begin{itemize}
  \item 処理は順番に処理される.
  \item 前の処理が終わるまでクライアントは待たされる.
  \end{itemize}
  \vfill
\end{frame}

%=========================================================================
\begin{frame}
  \frametitle{スレッドの役割(2)}
  マルチプログラミングを用いないWebサーバ \\
  \vfill
  \fig{scale=0.6}{singleProcMultiClient-crop.pdf}
  \vfill
  \begin{itemize}
  \item 工夫すると並列して処理することも可能
  \item しかし,プログラミングが難しい.
  \item しかし,SMPが活かせない.
  \end{itemize}
  \vfill
\end{frame}

%=========================================================================
\begin{frame}
  \frametitle{スレッドの役割(3)}
  マルチプログラミングを用いるWebサーバ(プロセス版) \\
  \vfill
  \fig{scale=0.6}{multiProc-crop.pdf}
  \vfill
  \begin{itemize}
  \item クライアント毎にプロセスを起動(\texttt{fork()})する.
  \item プログラミングは易しい.
  \item しかし,処理が重いし,プロセス間の情報共有の効率が悪い.
  \end{itemize}
  \vfill
\end{frame}

%=========================================================================
\begin{frame}
  \frametitle{スレッドの役割(4)}
  マルチプログラミングを用いるWebサーバ(スレッド版) \\
  \vfill
  \fig{scale=0.6}{multiThread-crop.pdf}
  \vfill
  \begin{itemize}
  \item クライアント毎にスレッドを起動する.
  \item プロセスの起動より10〜100倍速い.
  \item スレッド間は情報を共有しやすい.
  \item プログラミングは少し難しい.
  \end{itemize}
  \vfill
\end{frame}

%=========================================================================
\begin{frame}
  \frametitle{スレッドの形式(1) --カーネルスレッド--}
  \vfill
  \fig{scale=0.45}{kernelThread-crop.pdf}
  \vfill
  \begin{itemize}
  \item プロセスが複数の仮想CPUを持つ.
  \end{itemize}
  \vfill
\end{frame}

%=========================================================================
\begin{frame}
  \frametitle{スレッドの形式(2)--ユーザスレッド--}
  \vfill
  \fig{scale=0.45}{userThread-crop.pdf}
  \vfill
  \begin{itemize}
  \item ユーザプログラム(ライブラリ)の工夫でスレッドを実現する.
  \item 並行(Parallel)実行はできない.
  \item どれかのスレッドがブロックすると全スレッドが停止する.
  \end{itemize}
  \vfill
\end{frame}

%=========================================================================
\begin{frame}
  \frametitle{スレッドの形式(3)--スレッドモデル--}
  上記の2方式を組み合わせた3種類のスレッドモデルがある.
  \vfill
  \begin{itemize}
  \item \emph{One-to-One Model}   \\
    一つのスレッドを一つのカーネルスレッドで実行する.
  \item \emph{Many-to-One Model}  \\
    複数のスレッドを一つのカーネルスレッドで実行する.
  \item \emph{Many-to-Many Model} \\
    複数のスレッドを複数のカーネルスレッドで実行する.
  \end{itemize}
  \vfill
\end{frame}

%=========================================================================
\begin{frame}
  \frametitle{スレッドの使用例(1)}
  M個のスレッドで手分けをして合計を計算する様子 \\
  (複数のカーネルスレッド(CPU)で手分けすることで短時間で処理が終わるはず)
  \vfill
  \fig{scale=0.6}{threadedSum-crop.pdf}
\end{frame}

%=========================================================================
\begin{frame}
  \frametitle{スレッドの使用例(2)}
  M個のスレッドで合計を計算するプログラム \\
  \vfill
  \smp{language={C},lastline=21,frame=tlr,
    numbers=left,xleftmargin=5mm,firstnumber=1}{pThread/threadTest.c}
  \vfill
\end{frame}

%=========================================================================
\begin{frame}
  \frametitle{スレッドの使用例(3)}
  \smp{language={C},firstline=22,frame=lrb,
    numbers=left,xleftmargin=5mm,firstnumber=22}{pThread/threadTest.c}
  \vfill
\end{frame}

%=========================================================================
\begin{frame}
  \frametitle{スレッドの使用例(4)}
  実行時間の計測結果 \\
  \vfill
  \tbl{scale=0.7}{threadTimeTbl.pdf}
  \vfill
  \tbl{scale=0.8}{threadTimeGrph.pdf}
  \vfill
  6コアのMac Proで計測\\
  (Hyper-Threadingのお陰で6コアと12コアの中間的な振舞)
  \vfill
\end{frame}

%=========================================================================
\section{TacOSのCPU仮想化}
\begin{frame}
  \frametitle{実装例}
  \vfill
  \begin{center}
    \textbf{\LARGE 第19章} \\
    \textbf{\Huge TacOSのCPU仮想化}\\
    \vfill
    \fig{scale=0.40}{tacosOrganization-crop.pdf}
  \end{center}
  \vfill
\end{frame}

%=========================================================================
\begin{frame}
  \frametitle{TacOSのPCB}
  \begin{itemize}
    \item 仮想CPU(\texttt{sp})
    \item プロセス番号(\texttt{pid})
    \item 状態(\texttt{stat})
    \item 優先度(\texttt{nice}, \texttt{enice})
    \item プロセステーブルのインデクス(\texttt{idx})
    \item イベント用カウンタとセマフォ(\texttt{evtCnt,evtSem})
    \item プロセスのアドレス空間(\texttt{memBase},\texttt{memLen})
    \item プロセスの親子関係の情報(\texttt{parent},\texttt{exitStat})
    \item オープン中のファイル一覧(\texttt{fds[]})
    \item PCBリストの管理(\texttt{prev},\texttt{next})
    \item スタックオーバーフローの検知(\texttt{magic})
  \end{itemize}
  \vfill
\end{frame}

%=========================================================================
\begin{frame}
  \frametitle{TacOSのPCB(前半)}
  \src{firstline=56,lastline=67,frame=tlr,xleftmargin=5mm}{kernel/process.hmm}
  \vspace{-5mm}\src{firstline=68,lastline=76,frame=lr,
  numbers=left,xleftmargin=5mm,firstnumber=1}{kernel/process.hmm}
  \vfill
\end{frame}

%=========================================================================
\begin{frame}
  \frametitle{TacOSのPCB(後半)}
  \src{firstline=77,lastline=96,frame=lrb,
  numbers=left,xleftmargin=5mm,firstnumber=10}{kernel/process.hmm}
  \vfill
\end{frame}

%=========================================================================
\begin{frame}
  \frametitle{TacOSの実行可能列}
  \begin{itemize}
    \item PCBの双方向環状リスト
    \item 優先度順にソート(\texttt{curProc}は実行中のプロセス)
    \item 末尾に\texttt{idle}プロセスが常駐 \\
      \vfill
      \fig{scale=0.5}{tacosReadyQueue-crop.pdf}
  \end{itemize}
  \vfill
\end{frame}

%=========================================================================
\begin{frame}
  \frametitle{TacOSのメモリ配置}
  \fig{scale=0.4}{tacosMemMap-crop.pdf}
\end{frame}

%=========================================================================
\begin{frame}
  \frametitle{TacOSのタイマー管理プログラム}
  \src{firstline=465,lastline=483,
  numbers=left,xleftmargin=5mm,firstnumber=3}{kernel/kernel.cmm}
\end{frame}

%=========================================================================
\begin{frame}[fragile]
  \frametitle{参考(普通の関数)}
  \begin{minipage}{0.50\columnwidth}
    \|C--|言語の\|void|型関数
    \smp{%firstline=465,lastline=483,firstnumber=3,
        numbers=left,xleftmargin=5mm}{cmm/stdfunc.cmm}
  \end{minipage}
  \hfill
  \begin{minipage}{0.38\columnwidth}
    \smp{%firstline=465,lastline=483,firstnumber=3,
        numbers=left,xleftmargin=5mm}{cmm/stdfunc.s}
  \end{minipage}
\end{frame}

%=========================================================================
\begin{frame}[fragile]
  \frametitle{参考(interrupt 関数)}
  \begin{minipage}{0.50\columnwidth}
    \|C--|言語の\|interrupt|型関数
    \smp{%firstline=465,lastline=483,firstnumber=3,
        numbers=left,xleftmargin=5mm}{cmm/interrupt.cmm}
  \end{minipage}
  \hfill
  \begin{minipage}{0.38\columnwidth}
    \smp{%firstline=465,lastline=483,firstnumber=3,
        numbers=left,xleftmargin=5mm}{cmm/interrupt.s}
  \end{minipage}
\end{frame}

%=========================================================================
\begin{frame}[fragile]
  \frametitle{TacOSのコンテキスト保存プログラム(yield())}
  \src{firstline=49,lastline=56,frame=tlr,
  numbers=left,xleftmargin=5mm,firstnumber=1}{kernel/dispatcher.s}
  \vspace{-5mm}\begin{lstlisting}[frame=lr,xleftmargin=5mm]

        ...
  \end{lstlisting}
  \vspace{-5mm}\src{firstline=65,lastline=77,frame=lr,
  numbers=left,xleftmargin=5mm,firstnumber=17}{kernel/dispatcher.s}
\end{frame}

%=========================================================================
\begin{frame}[fragile]
  \frametitle{TacOSのコンテスト復旧プログラム(dispatch())}
  \|_dispatch|は\|_yield|(28行)の直後にある.(\textbf{つながっている!})
  \src{firstline=78,lastline=90,frame=lr,
    numbers=left,xleftmargin=5mm,firstnumber=1}{kernel/dispatcher.s}
  \vspace{-5mm}\begin{lstlisting}[frame=lr,xleftmargin=5mm]

        ...
  \end{lstlisting}
  \vspace{-5mm}\src{firstline=98,lastline=102,frame=lrb,
    numbers=left,xleftmargin=5mm,firstnumber=21}{kernel/dispatcher.s}
\end{frame}

%=========================================================================
%\section{練習問題}
%begin{frame}
%  \frametitle{練習問題}
%  \vfill
%  \begin{center}
%    \textbf{\Huge 練習問題}
%  \end{center}
%  \vfill
%\end{frame}

%=========================================================================
\begin{frame}[fragile]
  \frametitle{練習問題}
  \begin{itemize}
  \item 次の言葉の意味を説明しなさい.
    \begin{itemize}
    \item 時分割多重
    \item コンテキストスイッチ
    \item Dispatch(ディスパッチ)
    \item Preemption(プリエンプション)
    \item プロセスの状態
    \item プロセスの状態遷移
    \item RETI命令
    \item PCB
    \item 待ち行列
    \item 実行可能列
    \item スレッド
    \item カーネルスレッド
    \item ユーザスレッド
    \item One-to-One Model
    \item Many-to-One Model
    \item Many-to-Many Model
    \end{itemize}
  \item POSIXスレッドについて調査しなさい.
  \end{itemize}
\end{frame}

\end{document}

%=========================================================================
\begin{frame}[fragile]
  \frametitle{まとめ}
  \begin{itemize}
  \item 時分割多重によるCPUの仮想化
    \begin{itemize}
    \item プロセス毎にCPUコンテキストを持つ(仮想CPU)
    \item プロセスはいくつかの状態を持つ(Ready/Running/Wating)
    \item 実行できない状態のプロセスにはCPUを割り当てない.
    \item プロセスの状態はイベントやタイムスライシングにより変化する.
    \item プロセスを切換えることをコンテキストスイッチと言う.
    \item PCBはプロセスを表現する重要なカーネル内データ構造である.
    \item プロセスに複数のスレッドを導入するとプロセスがSMPになる.
    \item スレッドにはカーネルスレッド,ユーザスレッドがある.
    \end{itemize}
  \item TacOSのCPU仮想化
    \begin{itemize}
    \item PCB,メモリ配置
    \item コンテキスト保存プログラム(yield())
    \item コンテキスト復旧プログラム(dispatch())
    \end{itemize}
  \end{itemize}
\end{frame}
