%\documentclass[dvipdfmx]{beamer}
\documentclass[unicode,handout]{beamer}                   % lualatex の場合
\usepackage{mySld}

\begin{document}
\title[デッドロック]
      {オペレーティングシステム\\番外 デッドロック}
\date{}
\begin{frame}
  \titlepage
  \centerline{\url{https://github.com/tctsigemura/OSTextBook}}
\end{frame}

%=========================================================================
%\begin{frame}
%  \frametitle
%  \tableofcontents
%\end{frame}

%=========================================================================
\section{概要}
\begin{frame}
  \frametitle{デッドロック(deadlock)とは}
  資源を確保したい複数のプロセス(スレッド)があるとき. \\
  資源が解放されるのを待ち合う状態.
  \vfill
  \fig{scale=0.6}{deadlock-crop.pdf}
  \vfill
\end{frame}

%=========================================================================
\begin{frame}
  \frametitle{デッドロックが発生する条件}
  デッドロックが発生するための必要条件
  \begin{enumerate}
    \item[(1)] 相互排除(資源が排他的に利用される)
    \item[(2)] 確保待ち(資源を確保した状態で待つ)
    \item[(3)] 横取り不可(使用中の資源を横取りできない)
    \item[(4)] 循環待ち(待ちがループしている)
  \end{enumerate}
  (1),(3)は資源の種類によってはどうしようもない.
  \vfill
  プリンタは,相互排除,横取り不可な資源の例.
  \vfill
  \fig{scale=0.6}{deadlockPrinter-crop.pdf}
  \vfill
\end{frame}

%=========================================================================
\section{防止}
\begin{frame}
  \frametitle{デッドロックの発生を防止する方法(1)}
  前出の(2),(4)のどちらかを防止すればデッドロックは発生しない.\\
  \textbf{資源の一括確保(P\_AND)}  … (2)確保待ちにしない.
  \vfill
  \fig{scale=0.6}{deadlockPrevention1-crop.pdf}
  \vfill
  資源の利用率が悪くなる.
\end{frame}

%=========================================================================
\begin{frame}
  \frametitle{デッドロックの発生を防止する方法(2)}
  \textbf{資源の確保順序に制約}  … (4)循環待ちにしない.
  \vfill
  \fig{scale=0.6}{deadlockPrevention2-crop.pdf}
  \vfill
  不公平が生じるかも(順番付けできない例「食事する哲学者問題」).
\end{frame}

\end{document}
