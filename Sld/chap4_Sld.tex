%\documentclass[dvipdfmx]{beamer}
\documentclass[unicode]{beamer}                   % lualatex の場合
\usepackage{mySld}

\begin{document}
\title[スケジューリング]
      {オペレーティングシステム\\第4章 スケジューリング}
\date{}
\begin{frame}
  \titlepage
  \centerline{\url{https://github.com/tctsigemura/OSTextBook}}
\end{frame}

%=========================================================================
%\begin{frame}
%  \frametitle
%  \tableofcontents
%\end{frame}

%=========================================================================
\section{評価基準}
\begin{frame}
  \frametitle{評価基準}
  \begin{itemize}
    \item スループット(Throughput)
    \item ターンアラウンド時間(Turnaround time)
    \item レスポンス時間(Response time)
    \item 締め切り(Deadline)
    \item その他(公平性,省エネ,予測性など)
  \end{itemize}
\end{frame}

%=========================================================================
\section{システムごとの目標}
\begin{frame}
  \frametitle{スケジューリングの目標}
  \tbl{scale=1.0}{schedulingObjective.pdf}
\end{frame}

%=========================================================================
\section{プロセスの振舞}
\begin{frame}
  \frametitle{CPUバウンドプロセス}
  \fig{scale=0.7}{cpuBound-crop.pdf}
  \begin{itemize}
    \item 動画圧縮の例
    \item I/Oバウンドプロセス(エクセル)
  \end{itemize}
\end{frame}

%=========================================================================
\begin{frame}
  \frametitle{I/Oバウンドプロセス}
  \fig{scale=0.7}{ioBound-crop.pdf}
  \begin{itemize}
    \item スプレッドシートの例
  \end{itemize}
\end{frame}

%=========================================================================
\section{スケジューリング方式}
\subsection{FCFS}
\begin{frame}
  \frametitle{FCFSスケジューリング(1)}
  \small\begin{tabular}{c c c}
    プロセス & 到着時刻 & CPUバースト時間(ms) \\
    \hline
    $P_1$    & 0 & 100 \\
    $P_2$    & 0 & 20 \\
    $P_3$    & 0 & 10 \\
  \end{tabular}
  \gant{scale=0.8}{fcfs1.pdf}
  \begin{itemize}
    \item $P_1$,$P_2$,$P_3$の順に実行
    \item 平均ターンアラウンド時間($(100+120+130) / 3 = 117$ ms)
  \end{itemize}
\end{frame}

%=========================================================================
\begin{frame}
  \frametitle{FCFSスケジューリング(2)}
  \small\begin{tabular}{c c c}
    プロセス & 到着時刻 & CPUバースト時間(ms) \\
    \hline
    $P_1$    & 0 & 100 \\
    $P_2$    & 0 & 20 \\
    $P_3$    & 0 & 10 \\
  \end{tabular}
  \gant{scale=0.8}{fcfs2.pdf}
  \begin{itemize}
    \item $P_2$,$P_3$,$P_1$の順に実行
    \item 平均ターンアラウンド時間($(20+30+130) / 3 = 60$ ms)
  \end{itemize}
\end{frame}

%=========================================================================
\subsection{SJF}
\begin{frame}
  \frametitle{SJFスケジューリング}
  \small\begin{tabular}{c c c}
    プロセス & 到着時刻 & CPUバースト時間(ms) \\
    \hline
    $P_1$    & 0 & 100 \\
    $P_2$    & 0 & 20 \\
    $P_3$    & 0 & 10 \\
  \end{tabular}
  \gant{scale=0.8}{sjf1.pdf}
  \begin{itemize}
    \item 平均ターンアラウンド時間($(10+30+130) / 3 = 57$ ms)
  \end{itemize}
\end{frame}

%=========================================================================
\subsection{SRTF}
\begin{frame}
  \frametitle{SJFスケジューリング(比較のため)}
  \small\begin{tabular}{c c c}
    プロセス & 到着時刻 & CPUバースト時間(ms) \\
    \hline
    $P_1$    & 0  & 60 \\
    $P_2$    & 10 & 40 \\
    $P_3$    & 60 & 30 \\
  \end{tabular}
  \gant{scale=0.8}{sjf2.pdf}
  \begin{itemize}
    \item SJFはプリエンプションなし
    \item 平均ターンアラウンド時間\\
      ($((60-0)+(90-10)+(130-60))/3=70$ ms)
  \end{itemize}
\end{frame}

%=========================================================================
\begin{frame}
  \frametitle{SRTFスケジューリング}
  \small\begin{tabular}{c c c}
    プロセス & 到着時刻 & CPUバースト時間(ms) \\
    \hline
    $P_1$    & 0  & 60 \\
    $P_2$    & 10 & 40 \\
    $P_3$    & 60 & 30 \\
  \end{tabular}
  \gant{scale=0.8}{srtf1.pdf}
  \begin{itemize}
    \item SRTFはプリエンプションあり
    \item 平均ターンアラウンド時間\\
      ($((130-0)+(50-10)+(90-60))/3=67$ ms)
  \end{itemize}
\end{frame}

%=========================================================================
\subsection{RR}
\begin{frame}
  \frametitle{RRスケジューリング(1)}
  \small\begin{tabular}{c c c}
    プロセス & 到着時刻 & CPUバースト時間(ms) \\
    \hline
    $P_1$    & 0  & 60 \\
    $P_2$    & 10 & 40 \\
    $P_3$    & 60 & 30 \\
  \end{tabular}
  \gant{scale=0.8}{rr1.pdf}
  \begin{itemize}
    \item クォンタムタイム=10ms
    \item 平均ターンアラウンド時間\\
      ($((120-0)+(90-10)+(130-60))/3=90$)
  \end{itemize}
\end{frame}

%=========================================================================
\begin{frame}
  \frametitle{RRスケジューリング(2)}
  \small\begin{tabular}{c c c}
    プロセス & 到着時刻 & CPUバースト時間(ms) \\
    \hline
    $P_1$    & 0  & 60 \\
    $P_2$    & 10 & 40 \\
    $P_3$    & 60 & 30 \\
  \end{tabular}
  \gant{scale=0.8}{rr2.pdf}
  \begin{itemize}
    \item クォンタムタイム=50ms
    \item 平均ターンアラウンド時間\\
      ($((100-0)+(90-10)+(130-60))/3=83$ ms)
  \end{itemize}
\end{frame}

%=========================================================================
\subsection{優先度順}
\begin{frame}
  \frametitle{優先度順スケジューリング}
  \begin{itemize}
    \item 静的・動的
    \item スタベーション
    \item エージング
  \end{itemize}
\end{frame}

%=========================================================================
\subsection{FB}
\begin{frame}
  \frametitle{FBスケジューリング}
  \fig{scale=0.4}{multilevelFeedbackQueue-crop.pdf}
  \begin{itemize}
    \item エージング
  \end{itemize}
\end{frame}

%=========================================================================
\subsection{TacOSのスケジューラ}
\begin{frame}[fragile]
  \frametitle{TacOSのスケジューラ}
  \src{firstline=129,lastline=137,
    numbers=left,xleftmargin=5mm,firstnumber=1}{kernel/kernel.cmm}
\end{frame}

%=========================================================================
\begin{frame}
  \frametitle{TacOSの実行可能列(参考)}
  \begin{itemize}
    \item {\tt yield}
    \item {\tt dispatch}
    \item 実行可能列
      \fig{scale=0.5}{tacosReadyQueue-crop.pdf}
  \end{itemize}
\end{frame}

\end{document}
