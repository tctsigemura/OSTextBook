%\documentclass[dvipdfmx]{beamer}
\documentclass[unicode,handout]{beamer}                   % lualatex の場合
\usepackage{mySld}

\begin{document}
\title[スケジューリング]
      {オペレーティングシステム\\第4章 スケジューリング}
\date{}
\begin{frame}
  \titlepage
  \centerline{\url{https://github.com/tctsigemura/OSTextBook}}
\end{frame}

%=========================================================================
%\begin{frame}
%  \frametitle
%  \tableofcontents
%\end{frame}

%=========================================================================
\section{評価基準}
\begin{frame}
  \frametitle{評価基準}
  \begin{itemize}
    \item スループット(Throughput)
    \item ターンアラウンド時間(Turnaround time)
    \item レスポンス時間(Response time)
    \item 締め切り(Deadline)
    \item その他(公平性,省エネ,予測性など)
  \end{itemize}
\end{frame}

%=========================================================================
\section{システムごとの目標}
\begin{frame}
  \frametitle{システムごとの目標}
  \tbl{scale=1.0}{schedulingObjective.pdf}
\end{frame}

%=========================================================================
\section{プロセスの振舞}
\begin{frame}
  \frametitle{CPUバウンドプロセス}
  動画圧縮の例\\
  \vfill
  \fig{scale=0.7}{cpuBound-crop.pdf}
  \vfill
\end{frame}

%=========================================================================
\begin{frame}
  \frametitle{I/Oバウンドプロセス}
  スプレッドシートの例\\
  \vfill
  \fig{scale=0.7}{ioBound-crop.pdf}
  \vfill
\end{frame}

%=========================================================================
\section{スケジューリング方式}
\subsection{FCFS}
\begin{frame}
  \frametitle{FCFSスケジューリング(1)}
  FCFS(First-Come, First-Served)
  \vfill
  \begin{itemize}
    \item プリエンプションしないスケジューリング方式
  \end{itemize}
  \vfill
  \small\begin{tabular}{c c c}
    プロセス & 到着時刻 & CPUバースト時間(ms) \\
    \hline
    $P_1$    & 0 & 100 \\
    $P_2$    & 0 & 20 \\
    $P_3$    & 0 & 10 \\
  \end{tabular}
  \vfill
  \gant{scale=0.8}{fcfs1.pdf}
  \vfill
  \begin{itemize}
    \item $P_1$,$P_2$,$P_3$の順に実行
    \item 平均ターンアラウンド時間($(100+120+130) / 3 = 117$ ms)
    \item 最悪の平均ターンアラウンド時間を選択することもある.
  \end{itemize}
  \vfill
\end{frame}

%=========================================================================
\begin{frame}
  \frametitle{FCFSスケジューリング(2)}
  \begin{itemize}
    \item 平均ターンアラウンド時間は 到着順により大きく変化する.
  \end{itemize}
  \vfill
  \small\begin{tabular}{c c c}
    プロセス & 到着時刻 & CPUバースト時間(ms) \\
    \hline
    $P_1$    & 0 & 100 \\
    $P_2$    & 0 & 20 \\
    $P_3$    & 0 & 10 \\
  \end{tabular}
  \vfill
  \gant{scale=0.8}{fcfs2.pdf}
  \vfill
  \begin{itemize}
    \item $P_2$,$P_3$,$P_1$の順に実行
    \item 平均ターンアラウンド時間($(20+30+130) / 3 = 60$ ms)
  \end{itemize}
  \vfill
\end{frame}

%=========================================================================
\subsection{SJF}
\begin{frame}
  \frametitle{SJFスケジューリング}
  SJF(Shortest-Job-First)
  \vfill
  \begin{itemize}
    \item プリエンプションしないスケジューリング方式
  \end{itemize}
  \vfill
  \small\begin{tabular}{c c c}
    プロセス & 到着時刻 & CPUバースト時間(ms) \\
    \hline
    $P_1$    & 0 & 100 \\
    $P_2$    & 0 & 20 \\
    $P_3$    & 0 & 10 \\
  \end{tabular}
  \vfill
  \gant{scale=0.8}{sjf1.pdf}
  \vfill
  \begin{itemize}
    \item 平均ターンアラウンド時間($(10+30+130) / 3 = 57$ ms)
  \end{itemize}
  \vfill
\end{frame}

%=========================================================================
\subsection{SRTF}
\begin{frame}
  \frametitle{SRTFスケジューリング(1)}
  SRTF(Shortest-Remaining-Time-First)
  \vfill
  比較のためのSJFスケジューリングの例
  \vfill
  \small\begin{tabular}{c c c}
    プロセス & 到着時刻 & CPUバースト時間(ms) \\
    \hline
    $P_1$    & 0  & 60 \\
    $P_2$    & 10 & 40 \\
    $P_3$    & 60 & 30 \\
  \end{tabular}
  \vfill
  \gant{scale=0.8}{sjf2.pdf}
  \vfill
  \begin{itemize}
    \item SJFはプリエンプションなし
    \item 平均ターンアラウンド時間\\
      ($((60-0)+(90-10)+(130-60))/3=70$ ms)
  \end{itemize}
  \vfill
\end{frame}

%=========================================================================
\begin{frame}
  \frametitle{SRTFスケジューリング(2)}
  SRTF(Shortest-Remaining-Time-First)
  \vfill
  前のSJFと同じプロセスのをSRTFでスケジューリング
  \vfill
  \small\begin{tabular}{c c c}
    プロセス & 到着時刻 & CPUバースト時間(ms) \\
    \hline
    $P_1$    & 0  & 60 \\
    $P_2$    & 10 & 40 \\
    $P_3$    & 60 & 30 \\
  \end{tabular}
  \vfill
  \gant{scale=0.8}{srtf1.pdf}
  \vfill
  \begin{itemize}
    \item SRTFはプリエンプションあり
    \item 平均ターンアラウンド時間\\
      ($((130-0)+(50-10)+(90-60))/3=67$ ms)
  \end{itemize}
  \vfill
\end{frame}

%=========================================================================
\subsection{RR}
\begin{frame}
  \frametitle{RRスケジューリング(1)}
  RR(Round-Robin)
  \vfill
  クォンタムタイムまでプリエンプションしない.
  \vfill
  \small\begin{tabular}{c c c}
    プロセス & 到着時刻 & CPUバースト時間(ms) \\
    \hline
    $P_1$    & 0  & 60 \\
    $P_2$    & 10 & 40 \\
    $P_3$    & 60 & 30 \\
  \end{tabular}
  \vfill
  \gant{scale=0.8}{rr1.pdf}
  \vfill
  \begin{itemize}
    \item クォンタムタイム=10ms
    \item 平均ターンアラウンド時間\\
      ($((120-0)+(90-10)+(130-60))/3=90$ ms)
  \end{itemize}
  \vfill
\end{frame}

%=========================================================================
\begin{frame}
  \frametitle{RRスケジューリング(2)}
  \vfill
  \small\begin{tabular}{c c c}
    プロセス & 到着時刻 & CPUバースト時間(ms) \\
    \hline
    $P_1$    & 0  & 60 \\
    $P_2$    & 10 & 40 \\
    $P_3$    & 60 & 30 \\
  \end{tabular}
  \vfill
  \gant{scale=0.8}{rr2.pdf}
  \vfill
  \begin{itemize}
    \item クォンタムタイム=50ms
    \item 平均ターンアラウンド時間\\
      ($((100-0)+(90-10)+(130-60))/3=83$ ms)
  \end{itemize}
  \vfill
\end{frame}

%=========================================================================
\subsection{優先度順}
\begin{frame}
  \frametitle{優先度順スケジューリング}
  Priority
  \vfill
  \begin{itemize}
  \item 実行可能列を優先度順でソートしておく.
  \end{itemize}
  \vfill
  \fig{scale=0.35}{procQueue-crop.pdf}
  \vfill
  \begin{itemize}
    \item 静的優先度/動的優先度
    \item スタベーション(starvation):飢餓
    \item エージング(aging):老化,熟成
  \end{itemize}
  \vfill
\end{frame}

%=========================================================================
\subsection{FB}
\begin{frame}
  \frametitle{FBスケジューリング}
  FB(Multilevel Feedback Queue)\\
  \vfill
  \fig{scale=0.38}{multilevelFeedbackQueue-crop.pdf}
  \vfill
  \begin{itemize}
    \item エージング
  \end{itemize}
  \vfill
\end{frame}

%=========================================================================
\section{TacOSのスケジューラ}
\begin{frame}
  \frametitle{実装例}
  \vfill
  \begin{center}
    \textbf{\LARGE 第19章(6節)} \\
    \textbf{\Huge TacOSのスケジューラ}
  \end{center}
  \vfill
\end{frame}

%=========================================================================
\begin{frame}[fragile]
  \frametitle{TacOSのスケジューラ}
  \src{firstline=129,lastline=137,
    numbers=left,xleftmargin=5mm,firstnumber=1}{kernel/kernel.cmm}
  \vfill
\end{frame}

%=========================================================================
\begin{frame}
  \frametitle{TacOSの実行可能列(参考)}
  \begin{itemize}
  \item {\tt yield()}
  \item {\tt dispatch()}
    \vfill
  \item 実行可能列
    \fig{scale=0.5}{tacosReadyQueue-crop.pdf}
  \end{itemize}
  \vfill
\end{frame}

%=========================================================================
\section{練習問題}
\begin{frame}
  \frametitle{練習問題}
  \vfill
  \begin{center}
    \textbf{\Huge 練習問題}
  \end{center}
  \vfill
\end{frame}

%=========================================================================
\begin{frame}
  \frametitle{練習問題(1)}
  \begin{itemize}
  \item 次の言葉の意味を説明しなさい.
    \begin{itemize}
    \item スループット
    \item ターンアラウンド時間・レスポンス時間
    \item ハードリアルタイム・ソフトリアルタイム
    \item CPUバウンドプロセス・I/Oバウンドプロセス
    \item FCFSスケジューリング・SJFスケジューリング
    \item SRTFスケジューリング・RRスケジューリング
    \item 優先度順スケジューリング・FBケジューリング
    \item クォンタム時間
    \item スタベーション
    \item エージング
    \end{itemize}
  \end{itemize}
  \vfill
\end{frame}

%=========================================================================
\begin{frame}
  \frametitle{練習問題(2)}
  \begin{itemize}
  \item 次の三つのプロセスの実行順をガントチャートで示しなさい.
    また,平均ターンアラウンド時間を計算しなさい.
    \vfill
    \begin{center}
      \begin{tabular}{c c c}
        \emph{プロセス名} & \emph{到着時刻(ms)} &
        \emph{CPUバースト時間(ms)} \\\hline
        $P_1$  & 0  & 70 \\
        $P_2$  & 10 & 50 \\
        $P_3$  & 20 & 30
      \end{tabular}
    \end{center}
    \vfill
    \begin{itemize}
    \item FCFSでスケジューリングした場合
    \item SJFでスケジューリングした場合
    \item SRTFでスケジューリングした場合
    \item RR(但しクォンタム時間は20ms)でスケジューリングした場合
    \item RR(但しクォンタム時間は40ms)でスケジューリングした場合
    \item RR(但しクォンタム時間は60ms)でスケジューリングした場合
    \end{itemize}
  \end{itemize}
  \vfill
\end{frame}

\end{document}
