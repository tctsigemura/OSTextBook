%\documentclass[dvipdfmx]{beamer}
\documentclass[unicode]{beamer}                   % lualatex の場合
\usepackage{mySld}
\begin{document}
\title[ファイルシステムの概念]
      {オペレーティングシステム\\第14章 ファイルシステムの概念}
\date{}
\begin{frame}
  \titlepage
  \centerline{\url{https://github.com/tctsigemura/OSTextBook}}
\end{frame}

%\section{}
%=========================================================================
%\begin{frame}
%  \frametitle{}
%\end{frame}

\section{ファイルシステムの概念}
%=========================================================================
\begin{frame}
  \frametitle{ファイルシステム}
  \begin{itemize}
  \item ファイルシステムは二次記憶装置を
    \begin{itemize}
    \item 管理する.(どのセクタが,どのファイルの一部?)
    \item 抽象化する.(ハードディスク → ファイル)
    \item 仮想化する.(1台のハードディスク → 多数のファイル)
    \end{itemize}
  \item ファイルは一次元のバイト列(バイトストリーム) \\
    オペレーティングシステムはファイルの構造を決めない.
  \item ファイルは名前を持つ.
  \item 名前とバイト位置でデータが決まる.\\
    名前=ファイル名,バイト位置=ファイル内オフセット
  \end{itemize}
\end{frame}

%=========================================================================
\begin{frame}
  \frametitle{ファイルの名前付け}
  \fig{scale=0.6}{dirTree-crop.pdf}
  \begin{itemize}
  \item ファイルは木構造のディレクトリシステムに格納する.
  \item ディレクトリは名前とファイル本体のポインタを格納する.
  \item 階層構造を持った名前(\emph{パス})でファイルを特定する.
  \item \emph{絶対パス}はルートディレクトリを起点にする.
  \item \emph{相対パス}はワーキングディレクトリを起点にする.
  \end{itemize}

\end{frame}

%=========================================================================
\begin{frame}
  \frametitle{ファイルの別名(1)}
  \emph{別名があると便利な例}(最新のファイルはいつも同じ名前)
  \vfill
  ある日
  \begin{center}
    \begin{tabular}{l c l}
      \texttt{2017\_06\_30.log} &   & 2017年6月30日のファイル \\
      \texttt{2017\_07\_01.log} &   & 2017年7月1日のファイル  \\
      \texttt{2017\_07\_02.log} &   & 2017年7月2日のファイル  \\
      \texttt{today.log}        & → & \texttt{2017\_07\_02.log}
    \end{tabular}
  \end{center}
  \vfill
  次の日
  \begin{center}
    \begin{tabular}{l c l}
      \texttt{2017\_07\_01.log} &   & 2017年7月1日のファイル  \\
      \texttt{2017\_07\_02.log} &   & 2017年7月2日のファイル  \\
      \texttt{2017\_07\_03.log} &   & 2017年7月3日のファイル  \\
      \texttt{today.log}        & → & \texttt{2017\_07\_03.log}
    \end{tabular}
  \end{center}
  \vfill
\end{frame}

%=========================================================================
\begin{frame}
  \frametitle{ファイルの別名(2)}
  \begin{itemize}
  \item \emph{ハードリンク} \\
    \begin{itemize}
    \item ファイルシステムの仕組みとしてOSカーネルに組み込む.
    \item ファイル本体が複数のディレクトリ・エントリから指される.
    \item リンクカウントを用いる.
    \item ディレクトリをリンクするとループ検出が厄介 → 禁止!
    \end{itemize}
  \item \emph{シンボリックリンク} \\
    \begin{itemize}
    \item ファイルシステムの仕組みとしてOSカーネルに組み込む.
    \item 他ファイルのパスを格納した特別なファイル.
    \item リンク切れ状態が許される.(Webページのリンクに似ている)
    \end{itemize}
  \item \emph{ファイルシステムの外で実装されるリンク} \\
    \begin{itemize}
    \item Windowsのショートカット,macOSのエイリアスなど
    \item ファイルシステム本体が持つリンク機構は一定ではない. \\
      → 現代のOSは同時に複数のファイルシステムを使用する.\\
      → アプリに近い側でどのファイルシステムでも共通の仕組みを提供
    \end{itemize}
  \end{itemize}
\end{frame}

%=========================================================================
\begin{frame}
  \frametitle{ファイルの別名(3)}
  \emph{HFS+ファイルシステム上のmacOSのエイリアスの例}
  \lst{numbers=left,xleftmargin=5mm}{aliasOnHFS.txt}
  \begin{description}
    \item[3行] 拡張属性付きの通常ファイルとしてエイリアスが存在
    \item[4行] 拡張属性の名前は\texttt{com.apple.FinderInfo}
    \item[4行] 拡張属性のサイズは32バイト
  \end{description}
  ファイルシステムのより汎用的な機構である拡張属性を利用して,\\
  \emph{エイリアス}を実装している.
  \vfill
\end{frame}

%=========================================================================
\begin{frame}
  \frametitle{ファイルの別名(4)}
  \emph{FATファイルシステム上のmacOSのエイリアスの例}
  \lst{numbers=left,xleftmargin=5mm}{aliasOnFAT.txt}
  \begin{description}
    \item[4,5行] 拡張属性付きの通常ファイルとしてエイリアスが存在
    \item[2行]   隠しファイルができている!!
    \item[6行]   隠しファイルを消してみる.
    \item[9行]   拡張属性が消えてしまった!!
  \end{description}
  FATファイルシステムの規約の範囲で\emph{エイリアス}を実装している.
  \vfill
\end{frame}

%=========================================================================
\begin{frame}
  \frametitle{ボリュームのマウント}
  \begin{minipage}{0.49\columnwidth}
    \fig{scale=0.45}{mountTree-crop.pdf}
    \centerline{(a) マウント方式}
  \end{minipage}
  \begin{minipage}{0.49\columnwidth}
    \fig{scale=0.45}{mountDrive-crop.pdf}
    \centerline{(b) ドライブレター方式}
  \end{minipage}
  \vspace{5mm}
  \begin{itemize}
  \item 二つ目以降のボリュームの接続方法
  \item マウント方式
    \begin{itemize}
    \item ボリュームを既存のディレクトリに接続する.
    \item \texttt{/Volumes/NO NAME/hello.c}がUSBメモリのCプログラム
    \end{itemize}
  \item ドライブレター方式
    \begin{itemize}
    \item ボリュームを区別するドライブレターを用いる.
    \item \texttt{D:{\bs}hello.c}がUSBメモリのCプログラム
    \end{itemize}
  \end{itemize}
\end{frame}

%=========================================================================
\begin{frame}
  \frametitle{ファイルの属性(1)}
  \begin{itemize}
  \item \emph{名前}:ファイル名をファイルの属性と考える場合もある.
  \item \emph{識別子}:ファイル本体の番号など.
  \item \emph{型(タイプ)}:通常ファイル,ディレクトリ,リンクなど.
  \item \emph{保護}:\texttt{rwxrwxrwx}など.(後で詳しく)
  \item \emph{日時}:作成日時,最終変更日時など.
  \item \emph{所有者}:所有者,グループなど.
  \item \emph{位置}:ディスク上のどこにファイル本体があるか.\\
    (データを格納したブロック(セクタ)の番号など)
  \item \emph{サイズ}:ファイルのバイト数.
  \item \emph{拡張属性}:名前付きの小さな追加データ.\\
    ファイルシステムで用途を定めていない.
  \end{itemize}
\end{frame}

%=========================================================================
\begin{frame}
  \frametitle{ファイルの属性(2)}
  \lst{numbers=left,xleftmargin=5mm}{extendedAttr.txt}
  \vspace{-0.2cm}
  \begin{minipage}{0.71\columnwidth}
  \begin{description}
  \item[1行] 拡張属性付きでファイル一覧を表示.
  \item[4行] 拡張属性付きのファイルがある.
  \item[5行] 拡張属性の内容を表示してみる.
  \end{description}
  この例の拡張属性は,以下のようなものであった.
  \begin{itemize}
  \item 属性の名前:\texttt{com.apple.FinderInfo}
  \item 属性の大きさ:32バイト
  \item 意味:ファイルがエイリアスである.\\
    (ファイル本体がエイリアスのデータ)
  \end{itemize}
  \end{minipage}
  \begin{minipage}{0.27\columnwidth}
    \photo{scale=0.3}{alias.png}{}
  \end{minipage}
\end{frame}

%=========================================================================
\begin{frame}
  \frametitle{アクセス制御(1)}
  \begin{minipage}{0.54\columnwidth}
  ファイルの\emph{保護属性}に基づき,ファイルに誰が何をできるか制御する.
  \begin{itemize}
  \item \emph{ビット表現の保護モード}\\
    \begin{itemize}
    \item UNIXで使用される\texttt{rwxrwxrwx}のような情報.
    \item UNIXの場合,「所有者,グループ,その他」のユーザについて\\
      \begin{itemize}
      \item[r]:読める(Read),
      \item[w]:書ける(Write),
      \item[x]:実行できる(eXecute)
      \end{itemize}
      を指定する.
    \end{itemize}
  \end{itemize}
  \end{minipage}
  \begin{minipage}{0.44\columnwidth}
    \photo{scale=0.45}{extendedAttrs.png}{}
  \end{minipage}
\end{frame}

%=========================================================================
\begin{frame}
  \frametitle{アクセス制御(2)}
  \begin{itemize}
  \item \emph{ACL(Access Control List)}\\
    ファイル毎に,ユーザやグループを指定して細かな制御が可能

    \lst{numbers=left}{acl.txt}

    \begin{description}
    \item[1行] \texttt{a.txt}にACLが無いことを確認した.
    \item[3,4行] \texttt{chmod}コマンドで\texttt{a.txt}にACL追加した.
    \item[7,8行] 二行のACLが確認できる.
    \end{description}

    \begin{itemize}
    \item リストの先頭から順に評価する.
    \item 許可・不許可が決まったら評価を完了する.
    \item ACLで決まらない場合は\texttt{rwx}を使用する.
    \end{itemize}
  \end{itemize}
\end{frame}

%=========================================================================
\begin{frame}
  \frametitle{ファイルの種類}
  \begin{itemize}
  \item ファイルシステム(OSカーネル)で決まっている種類 \\
    (通常ファイル・ディレクトリ・リンクなど)
  \item アプリケーションなどが決めている種類 \\
    (通常ファイルの拡張子で区別する)
  \end{itemize}
  \tbl{scale=0.85}{filenameExtensions.pdf}
  \texttt{.app}だけはディレクトリの拡張子
\end{frame}

%=========================================================================
\begin{frame}
  \frametitle{ファイルシステムの操作(1)}
  \emph{ディレクトリ操作}
  \tbl{scale=0.9}{dirOperations.pdf}
  \begin{itemize}
  \item ファイルの作成はcreatシステムコールでもできる.
  \item ディレクトリの読み出しはライブラリ関数で行う.
  \item renameシステムコールはファイルの移動もできる.
  \end{itemize}
\end{frame}

%=========================================================================
\begin{frame}
  \frametitle{ファイルシステムの操作(2)}
  \emph{ファイル操作}
  \tbl{scale=0.9}{fileOperations.pdf}
  \begin{itemize}
  \item openはファイルの保護属性をチェックする.
  \item 切り詰めは専用のtruncateシステムコールも使える.
  \item ファイルの属性の読み書きができるべき.
  \end{itemize}
\end{frame}

%=========================================================================
\begin{frame}[fragile]
  \frametitle{ファイルシステムの操作(3)}
  \emph{ファイルの共有とロック}
\begin{lstlisting}[numbers=none]
  #include <sys/file.h>
  #define   LOCK_SH   1    // 共有ロック
  #define   LOCK_EX   2    // 排他ロック
  #define   LOCK_NB   4    // ブロックしない
  #define   LOCK_UN   8    // ロック解除
  int flock(int fd, int operation);
\end{lstlisting}
  \begin{itemize}
  \item \|LOCK_SH|:\emph{共有ロック(shared lock)}
  \item \|LOCK_EX|:\emph{排他ロック(exclusive lock)}
  \item \|LOCK_NB|:ロックできない時,ブロックしないでエラー
  \item openシステムコールにもロックの機能がある.
  \end{itemize}
  \vfill

  \emph{ワーキングディレクトリの変更}
\begin{lstlisting}[numbers=none]
  #include <unistd.h>
  int chdir(const char *path);
\end{lstlisting}

  \vfill
\end{frame}

%=========================================================================
\begin{frame}
  \frametitle{ファイルシステムの健全性(1)}
  \emph{一貫性チェック}
  \begin{itemize}
  \item 正常終了時にはファイルシステムにアンマウントの印をする.
  \item OSの起動時に印がなかったら一貫性チェックをする.
  \item \emph{メタデータ}の矛盾を解消するだけ.
  \item ファイルが消えたり,データが消えたりは修復できない.
  \end{itemize}
  \vfill
\end{frame}

%=========================================================================
\begin{frame}
  \frametitle{ファイルシステムの健全性(2)}
  \emph{ジャーナリング・ファイルシステム}
  \vspace{3mm}\fig{scale=0.45}{journaling-crop.pdf}
  \begin{itemize}
  \item データベースのWAL(Write Ahead Logging)のアイデア.
  \item NTFS,ext3,ext4,HFS+ 等が該当する.
  \end{itemize}
  \vfill
\end{frame}

%=========================================================================
\begin{frame}
  \frametitle{練習問題(1)}
  \begin{enumerate}
  \item[1.] 次の言葉の意味を説明しなさい.
    \begin{itemize}
    \item ディレクトリシステム
    \item パス,絶対パス,相対パス
    \item ディレクトリ,ファイル
    \item ハードリンク,シンボリックリンク
    \item ショートカット,エイリアス
    \item マウント,ドライブレター
    \item 拡張属性,ACL
    \end{itemize}

  \item[2.] 自分のオペレーティングシステムについて調査しなさい.\\
    (GUIよりCLIのコマンドを用い方がより詳しい観察ができる.)
    \begin{itemize}
    \item ショートカット(Windows),エイリアス(macOS)
    \item ファイルの属性(保護,日時,所有者,サイズ等)
    \item 拡張属性が使用できるオペレーティングシステムか?
    \item ACLが使用できるオペレーティングシステムか?
    \item USBメモリにはどのようなパスで到達できるか?
    \item ファイルシステムの一貫性をチェックするコマンドは何か?
    \end{itemize}
  \end{enumerate}
\end{frame}

%=========================================================================
\begin{frame}[fragile]
  \frametitle{練習問題(2)}
  \begin{enumerate}
  \item[3.] 自分が使用しているオペレーティングシステムで試してみなさい.
    \begin{itemize}
    \item ショートカットやエイリアスを作成し試してみなさい.
\begin{lstlisting}
      # macOSの場合の実行例
      $ echo aaa > a.txt
      $ open a.txt
      $ open a.txtのエイリアス      <--- エイリアスはGUIで作る
      $ cat a.txt
      $ cat a.txtのエイリアス
\end{lstlisting} %$
    \item UNIXやmacOSで実行して結果が異なる理由を考察しなさい.
\begin{lstlisting}
      # ハードリンクの場合          # シンボリックリンクの場合
      $ echo aaa > a.txt          $ echo aaa > a.txt
      $ echo bbb > b.txt          $ echo bbb > b.txt
      $ ln a.txt c.txt            $ ln -s a.txt c.txt
      $ mv a.txt d.txt            $ mv a.txt d.txt
      $ mv b.txt a.txt            $ mv b.txt a.txt
      $ cat c.txt                 $ cat c.txt
\end{lstlisting} %$
    \item ショートカットやエイリアスの振る舞いを調べる.\\
      (リンク先ファイルを削除・移動・別ファイルに置換えした場合など)
    \item ACLの追加・削除とその効果を確認する.
    \end{itemize}
  \end{enumerate}
\end{frame}

%=========================================================================
%\begin{frame}
%  \frametitle{}
%\end{frame}

\end{document}
