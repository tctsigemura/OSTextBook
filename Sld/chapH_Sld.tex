%\documentclass[dvipdfmx]{beamer}
\documentclass[handout]{beamer}                   % lualatex の場合
\usepackage{mySld}
\newcommand{\inode}{\texttt{i-node}}
\begin{document}
\title[ZFS]
      {オペレーティングシステム\\第17章 ZFS}
\date{}
\begin{frame}
  \titlepage
  \centerline{\url{https://github.com/tctsigemura/OSTextBook}}
\end{frame}

%\section{}
%=========================================================================
%\begin{frame}
%  \frametitle{}
%\end{frame}

\section{ZFS}
%=========================================================================
\begin{frame}
  \frametitle{ZFSの特徴}
  \begin{itemize}
  \item 2005年にサン・マイクロがOpenSolarisに実装して公開
  \item オープンソースで開発が続いている(FreeBSD等でも使用可)
  \item 大きな主記憶,高速マルチプロセッサが前提(8GiB,64bit CPU)
  \item \emph{COW(Copy On Write)}でデバイスに書き込む.\\
    (デバイスのブロックを上書きしないので前状態を維持)
  \item Uberblockを書き込みと同時に新しい状態になる.\\
    (Uberblockはファイルシステム管理データの根)
  \item チェックサムにより高い信頼性を確保(次ページ)
  \item 一瞬で\emph{スナップショット}や\emph{クローン}を作成 \\
    (COW の手法を使用し違いの出たブロックだけコピーし変更)
  \item ボリュームの代わりにストレージプールを使用(後のページ)
  \item ファイルサイズ等の制約が事実上無い($Zetta = 2^{70}$ )\\
    (ファイルサイズ$2^{64}B$, ストレージプールサイズ$2^{70}B$)
  \item ミラーやRAID-Z等が装備され信頼性・可用性が向上
  \item データ圧縮,重複除去機能があり(ディスクI/Oを少なくする)
  \end{itemize}
  \vfill
\end{frame}

%=========================================================================
\begin{frame}
  \frametitle{全ブロックにわたるチェックサム}
  \fig{scale=0.5}{zfsCheckSum-crop.pdf}
  \vfill
  \begin{itemize}
    \item ブロックポインタがチェックサムを持つ.
    \item メタデータだけでなく普通のデータにもチェックサムあり.
    \item チェックサムの不整合が見つかった場合,
      データの2重化(ミラー)がされていれば,
      自動的にミラーからデータを修復する.
  \end{itemize}
  \vfill
\end{frame}

%=========================================================================
\begin{frame}[fragile]
  \frametitle{従来の方式のボリューム}
  \fig{scale=0.6}{zfsVolume-crop.pdf}
  \vfill
  \begin{itemize}
  \item ファイルシステムの初期化以前にボリュームを決定し,
  \item 後でサイズの変更などはできない.
  \end{itemize}
  \vfill
\end{frame}
  
%=========================================================================
\begin{frame}[fragile]
  \frametitle{ZFSのストレージプール}
  \fig{scale=0.6}{zfsPool-crop.pdf}
  \vfill
  \begin{itemize}
  \item ストレージプールは沢山のデバイスを収容する.
  \item ファイルシステムからの要求に応じてブロックを割り付ける.
  \item C言語プログラムの\|malloc()|や\|free()|に似ている.
  \item ストレージプールに後でデバイスを追加することも可能.
  \end{itemize}
  \vfill
\end{frame}
  
%=========================================================================
\begin{frame}[fragile]
  \frametitle{ZFSのソフトウェア構成(1)}
  \fig{scale=0.40}{zfsSoftModule-crop.pdf}
  \vfill
  \begin{enumerate}
  \item[1.] システムコールは,OSカーネル本体がVNODE操作に変換する.
  \item[2.] ZPLは VNODE 操作をZFSのトランザクションに変換する.
  \item[3.] DMUは複数トランザクションを
    \emph{トランザショングループ}にする.
  \item[4.] SPAは,DMUがトランザクショングループをキャッシュに書込み終わると,
    キャッシュの内容をデバイスに反映させる.
  \end{enumerate}  
  \vfill
\end{frame}

%=========================================================================
\begin{frame}[fragile]
  \frametitle{ZFSのソフトウェア構成(2)}
  \fig{scale=0.45}{zfsTranscation-crop.pdf}
  \vfill
  \begin{enumerate}
  \item[3.] DMUは複数トランザクションを
    \emph{トランザショングループ}にする.
  \item[4.] SPAは,DMUがトランザクショングループをキャッシュに書込み終わると,
    キャッシュの内容をデバイスに反映させる.(バースト)
  \end{enumerate}  
  \vfill
\end{frame}

%=========================================================================
\begin{frame}[fragile]
  \frametitle{ストレージプールの構造(概要)}
  \centerline{
  \begin{minipage}{0.45\columnwidth}
    デバイス内部の構造
    \fig{scale=0.6}{zfsDevice.pdf}
  \end{minipage}
  \begin{minipage}{0.45\columnwidth}
    ボリュームラベル
    \fig{scale=0.6}{zfsVolumeLabel.pdf}
  \end{minipage}
  }
  \vfill
  \begin{itemize}
  \item デバイス(ディスク)の4箇所に同じボリュームラベルを書く.
  \item ボリュームラベルには128個のUberblockを格納できる.
  \item Uberblockはトランザクショングループ番号を含んでいる.
  \item Uberblockはトランザクショングループ番号を128で割った余りの位置に書く.
  \end{itemize}
  \vfill
\end{frame}

%=========================================================================
\begin{frame}[fragile]
  \frametitle{ストレージプールの更新(1)}
  \fig{scale=0.45,clip,trim=0 290 0 0}{zfsCommit-crop.pdf}
  \vfill
  \begin{enumerate}
    \item[1.] Uberblockを起点とする木構造でブロックが記録されている.
    \item[2.] 変更するには,新しいブロックを確保し内容を書き込む(COW).
  \end{enumerate}
  \vfill
\end{frame}

%=========================================================================
\begin{frame}[fragile]
  \frametitle{ストレージプールの更新(2)}
  \fig{scale=0.45,clip,trim=0 0 0 288}{zfsCommit-crop.pdf}
  \vfill
  \begin{enumerate}
    \item[3.] メタデータブロックもCOWで更新する.
    \item[4.] Uberblockを新しい領域に書き込む.\\
      (トランザクショングループ番号が最新のUberblockが有効)\\
      (古い世代のブロックは解放され,再利用される.)
  \end{enumerate}
  \vfill
\end{frame}

%=========================================================================
\begin{frame}[fragile]
  \frametitle{ブロックポインタ}
  図中でブロックを指していた\emph{矢印}を表現するデータ構造を
  \emph{ブロックポインタ}と呼ぶ.
  ブロックポインタはデータ多重化のために最大3組のアドレスを記録できる.
  ブロックポインタの内容は以下の通り.
  \vfill
  \begin{itemize}
  \item \emph{サイズ}:ブロックの大きさに関する情報
  \item \emph{チェックサム}(64ビット):ブロックのチェックサム(最大3個)
  \item \emph{ブロックのアドレス}:
    ブロックのストレージプール内での格納位置に関する情報(最大3個)
    (デバイス,デバイス内アドレス)
  \item \emph{タイムスタンプ}:
    ブロックを作成したトランザクショングループの番号
    (ブロックが削除される時にスナップショットと比較)
  \item \emph{その他}:チェックサム計算に使用するアルゴリズムの種類,
    データ圧縮に使用するアルゴリズムの種類,圧縮後のサイズなど...
  \end{itemize}
  \vfill
\end{frame}

%=========================================================================
\begin{frame}[fragile]
  \frametitle{Dnode(基本)}
  あらゆるオブジェクトを表現する512バイトのデータ構造である.
  USFの\texttt{i-node}に似ているが,
  ファイル,ディレクトリだけでなく,ファイルシステム,
  スナップショット,クローンなども表現する.
  \vfill
  \centerline{\fig{scale=0.55}{zfsDnodeDirect-crop.pdf}}
  \vfill
  \begin{itemize}
  \item dnodeは三つ以内のブロックポインタを格納することができる.
  \item dnodeは表現するオブジェクトに応じたデータを格納する領域を持っている.
    (この領域はブロックポインタと共用になっている)
  \end{itemize}
  \vfill
\end{frame}

%=========================================================================
\begin{frame}[fragile]
  \frametitle{Dnode(データが大きい場合)}
  \centerline{\fig{scale=0.55}{zfsDnodeIndirect-crop.pdf}}
  \vfill
  \begin{itemize}
  \item データの大きさが128KiB以内の場合は直接参照(前ページの図)
  \item 大きさが128KiBを超える場合は間接ブロック(上の図))
  \item 128KiBの間接ブロックはブロックポインタを最大1Ki個格納できる.
  \item $128KiB \times 1Ki = 128MiB$より大きなデーを表現する時は,
    多重の間接ブロック(最大6レベル)を用いる.($2^{64}$バイト以上)
  \end{itemize}
  \vfill
\end{frame}

%=========================================================================
\begin{frame}[fragile]
  \frametitle{ストレージプールの全体像}
  \fig{scale=0.4}{zfsStructure-crop.pdf}
  \vfill
\end{frame}

%=========================================================================
\begin{frame}[fragile]
  \frametitle{MOS (Meta Object Set) layer}
  \fig{scale=0.4,clip,trim=0 250 0 0}{zfsStructure-crop.pdf}
  \begin{itemize}
  \item Objsetはdnodeの配列を管理する.
  \item 配列には,ストレージプールプール全体の管理に関する情報や,
    ファイルシステムやスナップショット等の一覧が格納される.
  \item master node はストレージプールのコンフィグ,プロパティなど.
  \item space map はストレージプール内のブロックの割付を管理する.
  \end{itemize}
  \vfill
\end{frame}

%=========================================================================
\begin{frame}[fragile]
  \frametitle{ストレージプールの全体像(Object-set layer)}
  \fig{scale=0.4,clip,trim=0 0 0 250}{zfsStructure-crop.pdf}
  \begin{itemize}
  \item ファイルシステムはObjsetのdnode配列で表現する.
  \item dnodeリストがUFSの\texttt{i-node}リストに相当する.
  \item master node は root ディレクトリの dnode 番号などの情報.
  \item 図は,通常ファイルの例とディレクトリファイルの例
  \end{itemize}
  \vfill
\end{frame}

%=========================================================================
\begin{frame}[fragile]
  \frametitle{スナップショットの作成}
  \fig{scale=0.4,clip,trim=0 0 0 200}{zfsSnapshot-crop.pdf}
  \begin{itemize}
  \item ファイルシステムを表すObjsetのコピーを作る.
  \item MOS layer の dnode 配列に登録する.
  \item スナップショットは一瞬で作成できる.
  \item 変更がないデータブロックは共用する.
  \item 内容が書き換わるとCOWで必要最小限のブロックのコピーを作る.
  \end{itemize}
  \vfill
\end{frame}

%=========================================================================
\begin{frame}[fragile]
  \frametitle{スナップショットとdeadlist}
  \fig{scale=0.55}{zfsDeadlist1-crop.pdf}
  \begin{itemize}
    \item ファイルシステム(\|FileSystem|)は時間とともに変化する.
    \item ある瞬間の状態を保存したものがスナップショット.
    \item \|B#N|と続く線はブロックとブロックの寿命を表す.
    \item \|Snapshot#N|はスナップショットを表す.
    \item \|deadlist|は使用されなくなったが解放できないブロックのリスト
  \end{itemize}
  \vfill
\end{frame}

%=========================================================================
\begin{frame}[fragile]
  \frametitle{ブロックの解放(1)}
  \fig{scale=0.55}{zfsDeadlist1-crop.pdf}
  \begin{itemize}
    \item \|B#1|は\|Snapshot#1|で使用されているので解放できない.
    \item \|B#2|と\|B#3|もスナップショットで使用されているので解放できない.
    \item \|B#4|はスナップショットで使用されていないので解放できる.
    \item 解放できないブロックは\|deadlist|に格納される.
  \end{itemize}
  \vfill
\end{frame}

%=========================================================================
\begin{frame}[fragile]
  \frametitle{ブロックの解放(2)}
  \fig{scale=0.55}{zfsDeadlist2-crop.pdf}
  \begin{itemize}
    \item \|Snapshot#2|を削除した状態.
    \item \|B#1|と\|B#2|は\|Snapshot#1|で使用されているので解放できない.
    \item \|B#3|は\|Snapshot#2|が削除されたので解放できる.
    \item \|B#1|と\|B#2|は\|FileSystem|の\|deadlist|に格納される.
  \end{itemize}
  \vfill
\end{frame}

%=========================================================================
\begin{frame}[fragile]
  \frametitle{まとめ}
  ZFSは大きな主記憶と高速なCPUを前提に設計されている.\\
  以下の特徴を持つ.
  \begin{itemize}
  \item COW(Copy On Write)を用い既存のブロックを上書きしない.
  \item COWのお蔭でシステムのクラッシュでも壊れない構造を実現.
  \item デバイス上の全てのデータについてチェックサムを持つ.
  \item スナップショットやクローンを一瞬で作ることができる.
  \item ボリュームの代わりにストレージプールを使用する.
  \item ストレージプールに後で新しいデバイスを追加可能.
  \end{itemize}
  \vfill
\end{frame}

%=========================================================================
\begin{frame}[fragile]
  \frametitle{練習問題(1)}
  次の言葉の意味を説明しなさい.
  分からない言葉は調べなさい.
  \begin{itemize}
  \item COW(Copy On Write)
  \item メタデータ
  \item チェックサム
  \item スナップショット
  \item クローン
  \item ボリューム
  \item ストレージプール
  \item ZPL(ZFS POSIX Layer)
  \item DMU(Data Management Unit)
  \item SPA(Storage Pool Allocator)
  \item VNODE
  \end{itemize}
  \vfill
\end{frame}

%=========================================================================
\begin{frame}[fragile]
  \frametitle{練習問題(2)}
  \begin{itemize}
  \item Uberblock
  \item ブロックポインタ
  \item Dnode
  \item MOS(Meta Object Set) layer
  \item Object-set layer
  \item space map
  \item master node
  \end{itemize}
  \vfill
\end{frame}

%=========================================================================
\begin{frame}[fragile]
  \frametitle{練習問題(3)}
  \begin{itemize}
  \item トランザクショングループ番号は64ビットです.
    毎秒100トランザクショングループを処理したとして,
    トランザクショングループ番号がオーバーフローするまでに約何年かかるか
    計算しなさい.
  \item ZFSで使用できるチェックサム計算アルゴリズムについて調べなさい.
  \item ファイルを表現するdnodeが間接レベル2の時,
    最大のファイルサイズは何バイトになるか計算しなさい.
  \item ブロックにリンクカウントを設け,
    スナップショットからの参照数を管理することで,
    ブロックの解放を判断するアイデアの利点と問題点を挙げなさい.
  \end{itemize}
  \vfill
\end{frame}

%=========================================================================
%\begin{frame}[fragile]
%  \frametitle{}
%  \vfill
%\end{frame}

\end{document}
