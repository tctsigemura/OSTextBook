%\documentclass[dvipdfmx]{beamer}
\documentclass[handout]{beamer}                   % lualatex の場合
\usepackage{mySld}

\begin{document}
\title[仮想記憶]{オペレーティングシステム\\第12章 仮想記憶}
\date{}
\begin{frame}
  \titlepage
  \centerline{\url{https://github.com/tctsigemura/OSTextBook}}
\end{frame}

%\section{}
%=========================================================================
%\begin{frame}
%  \frametitle{}
%\end{frame}

\section{仮想記憶}
%=========================================================================
\begin{frame}
  \frametitle{基本概念}
  ページングをベースに仮想記憶を実現する.
  \begin{itemize}
  \item システムの使用メモリ合計が物理メモリより大きい. → 実行可
  \item 単一のプログラムがメモリより大きい。 → 実行可
  \item ページテーブルのV=0を上手く使用する.
  \item V=0のページにアクセスするとページ不在割込み → OSへ
  \item ページテーブルのV=0に二つの場合がある.
    \begin{itemize}
      \item[1.] 無効な領域 → プロセス終了
      \item[2.] \emph{バッキングストア}に退避中 → 復旧して再開
    \end{itemize}
  \item プロセス生成時にバッキングストアにプロセスのイメージを作る.
  \item Windows,macOS,Linux等,現代のOSのほとんどが採用している.
  \end{itemize}
\end{frame}

%=========================================================================
\begin{frame}
  \frametitle{仮想記憶の基本}
  \fig{scale=0.45}{virtualMemoryBasic-crop.pdf}
\end{frame}

%=========================================================================
\begin{frame}
  \frametitle{デマンドページング(Demand Paging)}
  \fig{scale=0.45}{virtualMemoryWithZMagic-crop.pdf}
  \begin{itemize}
  \item ページをswap-inするための方式の一つ.
  \item 全てのページが不在の状態からスタートする.
  \item ページ不在を起こしたページをswap-inする.\\
    (使用しないページを読み込むような無駄が無い)
  \end{itemize}
\end{frame}

%=========================================================================
\begin{frame}
  \frametitle{プログラムファイルの直接swap-inによる実行}
  \fig{scale=0.45}{virtualMemoryWithZMagic-crop.pdf}
  \begin{itemize}
  \item デマンドページング用の実行可能形式ファイルを用いる.\\
    (このファイルはページサイズを意識した構造になっている)
  \item プログラムはファイルからswap-inする(R-Xに設定).
  \item 初期化データはファイルからswap-inする(RW-に設定).
  \item 非初期化データ,ヒープ,スタックはゼロにする(RW-に設定).
  \end{itemize}
\end{frame}

%=========================================================================
\begin{frame}
  \frametitle{プログラムのswap-out}
  \fig{scale=0.45}{virtualMemoryWithZMagic-crop.pdf}
  \begin{itemize}
  \item フレームが枯渇したら使用頻度の低いフレームを解放し再利用する.
  \item プログラム(R-X)は変化しないのでswap-outしない.
  \item 初期化データ(RW-)はバッキングストアにswap-outする.
  \item 非初期化データ,ヒープ,スタックもswap-outする.
  \end{itemize}
\end{frame}

%=========================================================================
\begin{frame}
  \frametitle{Copy on Write(1)}
  \fig{scale=0.40}{virtualMemoryFork-crop.pdf}
  \begin{itemize}
  \item fork-exec ではアドレス空間のコピーに無駄が多い. → vfork
  \item vforkは使いにくい.使いやすいforkを改良する.
  \item forkの後,書き込み可能ページを一時的に\texttt{R--}に設定しておく.
  \end{itemize}
\end{frame}

%=========================================================================
\begin{frame}
  \frametitle{Copy on Write(2)}
  \fig{scale=0.40}{virtualMemoryCOW-crop.pdf}
  \begin{itemize}
  \item 例えばスタックに\emph{書き込む}とメモリ保護違反割込みが発生する.
  \item この時点でOSが新しいフレームを割当て,内容を\emph{コピー}する.
  \item ページを\texttt{RW-}に変更しプロセスを再開する.
  \end{itemize}
\end{frame}

%=========================================================================
\begin{frame}
  \frametitle{メモリマップドファイル(1)}
  \fig{scale=0.45}{virtualMemoryWithMmap-crop.pdf}
  \begin{itemize}
  \item 仮想記憶機構を用いたファイルへのアクセス手段である.
  \item プロセスはメモリ上の配列のようにファイルにアクセスできる.
  \item ファイルアクセスで,一々システムコールを使用しない.\\
    (軽いファイルアクセス手段)
  \item 同じファイルを複数プロセスがマッピング → 共有メモリになる.
  \end{itemize}
\end{frame}

%=========================================================================
\begin{frame}[fragile]
  \frametitle{メモリマップドファイル(2)}
  UNIX のメモリマップドファイルの例(mmapシステムコール)

\begin{lstlisting}[numbers=none]
void * mmap(void *addr, size_t len, int prot, int flags, int fd, off_t offset);
\end{lstlisting}

  \begin{description}
  \item[\emph{戻り値}:] マップされた領域の先頭アドレスが返される.
  \item[\texttt{addr}:] マップしたい仮想アドレス空間の先頭アドレスを渡す.
  \item[\texttt{len}:] マップする領域の大きさを渡す.
  \item[\texttt{prot}:] 保護モード(protection:RWX)を表す値を渡す.
  \item[\texttt{flags}:]共用する(\|MAP_SHARED|)/
    しない(\|MAP_PRIVATE|)等
  \item[\texttt{fd}:] オープン済みファイルのファイルディスクリプタを渡す.
  \item[\texttt{offset}:] ファイル中のマッピング位置.
  \end{description}

  アドレスや長さはページサイズの整数倍にする.
\end{frame}

%=========================================================================
\begin{frame}
  \frametitle{メモリマップドファイル(3)}
  \smp{numbers=left,xleftmargin=5mm}{Mmap/mmapTest.c}
\end{frame}

%=========================================================================
\begin{frame}
  \frametitle{メモリマップドファイル(4)}
  メモリマップドファイルの仕組み
  \vspace{3mm}\fig{scale=0.45}{virtualMemoryWithMmap-crop.pdf}
  \begin{itemize}
  \item ファイルの読み込みはデマンドページングの要領で行う.
  \item ファイルの書き込みは
    \begin{itemize}
      \item Dirty ページを定期的にファイルに書き戻す.
      \item プロセスの終了やマッピングの解消時に書き戻す.
    \end{itemize}
  \end{itemize}
\end{frame}

%=========================================================================
\begin{frame}
  \frametitle{メモリマップドファイル(5)}
  read/writeシステムコールとの比較
  \vspace{3mm}\fig{scale=0.45}{mmapVsReadWrite-crop.pdf}
  \begin{itemize}
  \item ファイルを操作する度にシステムコールを発行する.\\
    (システムコールは重い処理)
  \item ディスクキャッシュとプログラムのバッファ間でメモリコピー\\
    (メモリコピーは重い処理)
  \end{itemize}
\end{frame}

%=========================================================================
\begin{frame}[fragile]
  \frametitle{メモリマップドファイル(6)}
  プロセスにローカルなマッピング
  \vspace{3mm}\fig{scale=0.45}{virtualMemoryWithPrivateMap-crop.pdf}
  \vspace{-5mm}\begin{itemize}
  \item これまでは\|MAP_SHARED|の例だった.
  \item \|MAP_PRIVATE|の例を紹介する.
  \item 最初は「ファイルデータ1」のように共有される(\texttt{R--}).
  \item 書き換えが発生した時点でコピーを作る(Copy on Write).
  \item 「ファイルデータ2」のようにプロセスは別々のコピーを参照する.
  \end{itemize}
\end{frame}

%=========================================================================
\begin{frame}[fragile]
  \frametitle{メモリマップドファイル(7)}
  無名メモリ
  \vspace{3mm}\fig{scale=0.45}{virtualMemoryWithAnonymousMap-crop.pdf}
  \begin{itemize}
  \item ファイルに関連付けられないがメモリ領域.
  \item \|MAP_ANON|フラグを用いる.
  \item アクセスがあった時点で作成される(デマンドページング).
  \item ページの初期値は\emph{ゼロ}.
  \end{itemize}
\end{frame}

%=========================================================================
\begin{frame}[fragile]
  \frametitle{メモリマップドファイル(8)}
  プログラムの実行とメモリマップドファイル
  \fig{scale=0.40}{virtualMemoryWithZMagic-crop.pdf}
  \begin{itemize}
  \item 実行形式ファイルをメモリにマッピングする.
  \item プログラムは,\|R-X|,\|MAP_SHARED/PRIVATE|でマッピングする.\\
    (プログラムはプロセス間で共用される)
  \item 初期化データは,\|RW-|,\|MAP_PRIVATE|でマッピングする.
  \item 非初期化データ,ヒープ,スタックは無名メモリ(\|RW-|,\|MAP_ANON|)
  \end{itemize}
\end{frame}

%=========================================================================
\begin{frame}[fragile]
  \frametitle{ページ置換えアルゴリズム}
  ページングによる仮想記憶で重要な三つのアルゴリズム
  \begin{enumerate}
  \item[1.]\emph{ページ読み込みアルゴリズム} :
    いつページをswap-inするか決める.
    普通は,既に学んだデマンドページングを用いる.
  \item[2.] \emph{ページ置き換えアルゴリズム} :
    フレーム不足時に,どのページを再利用するか決める.
  \item[3.] \emph{フレーム割付けアルゴリズム} :
    どのフレームを使用するか決める.
  \end{enumerate}
  \vfill
  ページ置き換えアルゴリズムが,
  将来,使用されないフレームをうまく選択しないと,
  swap-outしたページが直後にswap-inされることになり,
  システムの性能が著しく低下する.
\end{frame}

%=========================================================================
\begin{frame}
  \frametitle{局所性・ワーキングセット・フェーズ化(1)}
  \fig{scale=0.48}{locality-crop.pdf}
  \begin{description}
  \item[局所性]
    短い時間に着目すると,
    一部の連続したページが集中的にアクセスされる.
    → \emph{空間的局所性}\\
    あるページに着目すると一部の連続した時刻にアクセスが集中している.
    → \emph{時間的局所性}
  \end{description}
\end{frame}

%=========================================================================
\begin{frame}
  \frametitle{局所性・ワーキングセット・フェーズ化(2)}
  \fig{scale=0.48}{locality-crop.pdf}
  \begin{description}
  \item[ワーキングセット]
    ある時間にアクセスされるページの集合のこと.\\
    メモリに入り切らなくなると急激に性能が低下する.\\
    → \emph{スラッシング}
  \end{description}
\end{frame}

%=========================================================================
\begin{frame}
  \frametitle{局所性・ワーキングセット・フェーズ化(3)}
  \fig{scale=0.48}{locality-crop.pdf}
  \begin{description}
  \item[フェーズ化現象]
    ワーキングセットが急激に変化する現象のこと.\\
    「入力フェーズ」,「計算フェーズ」,「出力フェーズ」 \\
    フェーズ遷移時は局所性が失われる → ページ不在集中
\end{description}
\end{frame}

%=========================================================================
\begin{frame}
  \frametitle{LRU(Least Recently Used)アルゴリズム(1)}
  仮定:最近アクセスされていないページは,この先,アクセスされない.\\
  (時間的局所性があるなら最良な方法.)
  \vspace{3mm}\fig{scale=0.43}{pagingLRU-crop.pdf}
  \begin{enumerate}
  \item[1.] メモリアクセス毎にページテーブルにカウンタの値を書く.
  \item[2.] ページテーブルをスキャンし最も古いページを見つける.
  \item[3.] 見つけたページをswap-outし目的のページをswap-inする.(置換え)
  \end{enumerate}
\end{frame}

%=========================================================================
\begin{frame}
  \frametitle{LRU(Least Recently Used)アルゴリズム(2)}
  \fig{scale=0.43}{pagingLRU-crop.pdf}
  \vspace{3mm}問題点(LRUの完全な実装は困難と言われている)\\
  \begin{enumerate}
  \item[1.] ハードウェアのコスト
  \item[2.] ページ不在時の処理の重さ.(ページ不在は頻繁に発生)
  \end{enumerate}
  macOSのvm\_statで調べると毎秒数千回のページ不在!!
\end{frame}

%=========================================================================
\begin{frame}
  \frametitle{LFU(Least Frequently Used)アルゴリズム}
  NFU(Not Frequently Used)とも呼ばれる.
  \begin{itemize}
  \item LRUの近似方式の一種である.
  \item ページテーブルのRビットを使用.
  \item フレーム毎にカウンタを準備.
  \item 特別なハードウェアは不要.
  \end{itemize}
  \vfill
  \emph{アルゴリズム}
  \begin{enumerate}
  \item[1.] Rビットとフレームのカウンタをゼロにクリアする.
  \item[2.] 定期的(例えばTICK=20ms毎)にページテーブルをスキャンする.
    R=1のエントリを見つけたら対応するフレームのカウンタをインクリメントし,
    Rをゼロにクリアする.
  \item[3.] ページ不在時にフレームが不足したなら,
    カウンタの値が最小のフレームを置き換える.
  \end{enumerate}
\end{frame}

%=========================================================================
\begin{frame}
  \frametitle{エージングアルゴリズム}
  LFU(Least Frequently Used)アルゴリズムの改良\\
  \vfill
  \emph{LFUの問題点}\\
  一度カウンタの値が大きくなると,
  使用されなくなっても置き換えが起こらない.
  \vfill
  \emph{LFUの改良}\\
  定期的にページテーブルをスキャンする際の
  カウンタの更新方法を次のように改良する.
  \begin{description}
  \item[R=1のフレーム]
    $cnt \leftarrow cnt \div 2 + 0x8000$(カウンタは16bitと仮定)
  \item[R=0のフレーム]
    $cnt \leftarrow cnt \div 2$
  \end{description}
  この改良により,過去のRビットの影響が徐々に小さくなる.
\end{frame}

%=========================================================================
\begin{frame}
  \frametitle{FIFO(First-In First-Out)アルゴリズム}
  仮定:「長くメモリに滞在しているページは役割を終えている」\\
  特別なハードウェアを用いることなく,ソフトウェアだけで実現できる.
  \fig{scale=0.6}{pagingFIFO-crop.pdf}
  \vfill
  \emph{アルゴリズム}
  \begin{enumerate}
  \item[1.] swap-in したフレームをリストの最後に追加する.
  \item[2.] フレームが不足時は,リストの先頭のフレームを置き換える.
  \end{enumerate}
  \vfill
  ページテーブルのスキャンが不要なので非常に軽い.\\
  常時使用されるページも時間が経過するとswap-outされる問題がある.\\
  \emph{Beladyの異常な振る舞い}をすることがある.
\end{frame}

%=========================================================================
\begin{frame}
  \frametitle{Beladyの異常な振る舞いの例}
  \vfill
  FIFOアルゴリズムを用い,\\
  ページ参照ストリング(W : 1 2 3 4 1 2 5 1 2 3 4 5)の場合
  \begin{itemize}
  \item フレーム数(m=3)の場合(ページ不在9回)\\
    \vspace{3mm}\tbl{scale=1.0}{beladyAnomalyM3.pdf}
  \item フレーム数(m=4)の場合(ページ不在10回)\\
    \vspace{3mm}\tbl{scale=1.0}{beladyAnomalyM4.pdf}
  \end{itemize}
  メモリが多い方(m=4)のページ不在回数が多い.
  \vfill
\end{frame}

%=========================================================================
\begin{frame}
  \frametitle{Clock アルゴリズム}
  環状リストを用いる.Rビットも使用する.

  \fig{scale=0.55}{pagingClock-crop.pdf}

  \begin{enumerate}
  \item[1.] swap-inする度にフレームを環状リストに挿入していく.
  \item[2.] 定期的(例えばTICK=20ms毎)にRビットをクリアする.
  \item[3.] 時計の針が指しているフレームのRビットを調べる.
    \begin{description}
    \item[R=0の場合]
      ページは古い+最近アクセスされていない.→ 置換え
    \item[R=1の場合]
      ページは最近アクセスされている.→ 針を進める
    \end{description}
    最悪でも時計の針が一周回るとR=0のページが見つかる.
  \end{enumerate}
\end{frame}

%=========================================================================
\begin{frame}
  \frametitle{WSClock アルゴリズム(1)}

  \fig{scale=0.55}{pagingWSClock-crop.pdf}

  \begin{itemize}
  \item ワーキングセットを考慮したClockアルゴリズムである.
  \item 単純でパフォーマンスが良いので広く使用されている.
  \item アクセス時刻を記録した環状リストに用いる.
  \end{itemize}

\end{frame}

%=========================================================================
\begin{frame}
  \frametitle{WSClock アルゴリズム(2)}

  \fig{scale=0.55}{pagingWSClock-crop.pdf}

  \begin{itemize}
  \item 時刻が古くなっているフレームはワーキングセット外と判断.
  \item ページテーブルのRビットとDビットも使用する.
  \end{itemize}
\end{frame}

%=========================================================================
\begin{frame}
  \frametitle{WSClock アルゴリズム(3)}

  \fig{scale=0.55}{pagingWSClock-crop.pdf}

  \begin{enumerate}
  \item[1.] swap-inする度にフレームを環状リストに挿入していく.
  \item[2.] 定期的に全テーブルエントリのRビットをクリアする.\\
    その際,R=1だったフレームだけに現在時刻を記録する.
  \end{enumerate}
\end{frame}

%=========================================================================
\begin{frame}
  \frametitle{WSClock アルゴリズム(4)}

  \fig{scale=0.55}{pagingWSClock-crop.pdf}

  \begin{enumerate}
  \item[3.] フレーム不足なら時計の針のフレームを調べる.
    \begin{description}
    \item[R=1の場合]
      Rビットをクリアして次のフレームに進む.
    \item[時刻が新しい場合]
      次のフレームに進む.
    \item[時刻が古い場合]
      ページはワーキングセットに含まれていない.\\
      \emph{ページの置換え処理}を行う.
    \end{description}
  \end{enumerate}
\end{frame}

%=========================================================================
\begin{frame}
  \frametitle{WSClock アルゴリズム(5)}

  \fig{scale=0.55}{pagingWSClock-crop.pdf}

  \begin{enumerate}
  \item[3.] フレーム不足なら時計の針のフレームを調べる.
    \begin{description}
    \item[時刻が古い場合] \emph{ページの置換え処理}を行う.
      \begin{description}
      \item[D=1の場合]
        swap-outを予約し次のフレームに進む.
      \item[D=0の場合]
        このフレームを置き換える.
      \end{description}
    \end{description}
  \end{enumerate}
\end{frame}

%=========================================================================
\begin{frame}
  \frametitle{フレーム割り付けアルゴリズム(1)}
  \fig{scale=0.41}{hardBlock-crop.pdf}
  \begin{itemize}
  \item フレームはどれも均質
  \item SMPシステムであってもそうである.
  \end{itemize}
\end{frame}

%=========================================================================
\begin{frame}
  \frametitle{フレーム割り付けアルゴリズム(2)}
  \fig{scale=0.3}{intelServer-crop.pdf}
  \begin{itemize}
  \item 近年のサーバ用SMPでは均質ではない.
  \item メモリとCPUからなるノードを相互接続している.
  \item 自ノードと他ノードでメモリアクセス時間が異なる.
  \item このことを考慮したフレーム割付とCPUスケジューリングが必要.
  \end{itemize}
\end{frame}

%=========================================================================
%\begin{frame}
%  \frametitle{}
%\end{frame}

%=========================================================================
\begin{frame}
  \frametitle{練習問題}
  \begin{enumerate}
  \item[1.] 次の言葉の意味を説明しなさい.
    \begin{itemize}
    \item 仮想記憶
    \item デマンドページング
    \item swap-in,swap-out
    \item Copy on Write
    \item メモリマップドファイル
    \item 局所性
    \item ワーキングセット
    \item フェーズ化
    \item スラッシング
    \item ページ読み込みアルゴリズム
    \item ページ置き換えアルゴリズム
    \item ページ割付けアルゴリズム
    \item Beladyの異常な振る舞い
    \end{itemize}
  \end{enumerate}
\end{frame}

%=========================================================================
\begin{frame}
  \frametitle{練習問題}
  \begin{enumerate}
  \item[2.] 「Beladyの異常な振る舞いの例」で示した
    ページ参照ストリングとフレーム数を用い,
    他のページ置き換えアルゴリズムを適用した場合をトレースしなさい.
  \end{enumerate}
  \vfill
  \vfill
\end{frame}

\end{document}
