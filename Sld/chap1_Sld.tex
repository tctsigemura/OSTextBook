%\documentclass[dvipdfmx]{beamer}
\documentclass[unicode,handout]{beamer}                   % lualatex の場合
\usepackage{mySld}

\begin{document}
\title[オペレーティングシステムとは]
      {オペレーティングシステム\\第1章 オペレーティングシステムとは}
\date{}
\begin{frame}
  \titlepage
  \centerline{\url{https://github.com/tctsigemura/OSTextBook}}
\end{frame}

%=========================================================================
%\begin{frame}
%  \frametitle
%  \tableofcontents
%\end{frame}

%=========================================================================
\section{オペレーティングシステムとは}
\begin{frame}
  \frametitle{オペレーティングシステムとは}
  Windows,macOS,Linux,FreeBSD,Android,iOS などのこと.
  \begin{enumerate}
  \item[1.] カーネル(OSの本体)
  \item[2.] ライブラリ(プログラムが使用するサブルーチン,DLL)
  \item[3.] ユーザインタフェース(GUI,CLI)
  \item[4.] ユーティリティソフトウエア
    (ファイル操作,時計,シェル,システム管理 ...)
  \item[5.] プログラム開発環境
    (エディタ,コンパイラ,アセンブラ,リンカ,インタプリタ)
  \end{enumerate}

  \begin{itemize}
  \item \emph{広義}では上に列挙した全てのこと.
  \item \emph{狭義}では「カーネル」だけを指す.
  \end{itemize}
  この科目では,主に「カーネル」の仕組みを学ぶ.
  \vfill
\end{frame}

%=========================================================================
\section{オペレーティングシステムの役割}
\subsection{拡張マシンとしてのオペレーティングシステム}
\begin{frame}
  \frametitle{拡張マシンとしてのオペレーティングシステム}
  \fig{scale=0.6}{abstraction-crop.pdf}
  \begin{itemize}
    \item 抽象化(ファイル,プロセス)
    \item 拡張マシン(システムコール)
  \end{itemize}
\end{frame}

%=========================================================================
\subsection{ハードウェア管理プログラムとしてのオペレーティングシステム}
\begin{frame}
  \frametitle{ハードウェア管理プログラムとしての\\オペレーティングシステム}
  \fig{scale=0.5}{system-crop.pdf}
  \begin{itemize}
    \item 仮想化
    \item 時分割多重による仮想化
    \item 空間分割多重による仮想化
  \end{itemize}
  \vfill
\end{frame}

%=========================================================================
\section{オペレーティングシステムの歴史}
\subsection{第1世代(1945〜1955,黎明期)}
\begin{frame}
  \frametitle{第1世代}
  \begin{itemize}
    \item 真空管
    \item オペレーティングシステムなし
    \item 巨大TeC状態?
  \end{itemize}
  \vfill
\end{frame}

%=========================================================================
\subsection{第2世代(1955〜1965,トランジスタ)}
\begin{frame}
  \frametitle{第2世代(バッチ処理,その1)}
  \fig{scale=0.4}{batch-crop.pdf}
  \begin{itemize}
    \item メインフレーム
    \item バッチ,バッチ処理
    \item バッチモニタ
    \item JOB制御言語(JCL :  Job Control Language)
    \item 実行モード
    \item システムコール
    \item 記憶保護
  \end{itemize}
  \vfill
\end{frame}

%=========================================================================
\begin{frame}
  \frametitle{第2世代(バッチ処理,その2)}
  \photo{scale=0.15}{punchcard.jpg}{}
  \fig{scale=0.37}{job-crop.pdf}
  \begin{itemize}
    \item 紙カード
    \item JOBの構成
  \end{itemize}
  \vfill
\end{frame}

%=========================================================================
\subsection{第3世代(1966〜1980,ICとマルチプログラミング)}
\begin{frame}
  \frametitle{第3世代(メインフレーム)}
  \photo{scale=0.2}
   {Bundesarchiv_B_145_Bild-F038812-0014,_Wolfsburg,_VW_Autowerk.jpg}
      {\tiny
        フォルクスワーゲンで使われる System/360 \\
        ウィキメディア /
        Bundesarchiv, B 145 Bild-F038812-0014 /
        Schaack, Lothar / CC-BY-SA 3.0 de
      }
  \begin{itemize}
    \item IBM System/360,シリーズ化
    \item 仮想記憶
  \end{itemize}
  \vfill
\end{frame}

%=========================================================================
\begin{frame}
  \frametitle{第3世代(マルチプログラミング)}
  \fig{scale=0.7}{multiprogramming-crop.pdf}
  \begin{itemize}
    \item OS/360
    \item MULTICS(MULTiplexed Information and Computing Service)
    \item UNIX(ユニックス)
    \item Dynabook
  \end{itemize}
  \vfill
\end{frame}

%=========================================================================
\begin{frame}
  \frametitle{第3世代(タイムシェアリング : TSS)}
  \fig{scale=0.5}{timesharing-crop.pdf}
  \photo{scale=0.15}{724px-Televideo925Terminal.jpeg}
  {\tiny 写真:
      \url{http://commons.wikimedia.org/wiki/File:Televideo925Terminal.jpg}
(パブリックドメイン)}
  \begin{itemize}
    \item TSS(Time Sharing System)
    \item ターミナル
  \end{itemize}
  \vfill
\end{frame}

%=========================================================================
\begin{frame}
  \frametitle{次の世代に向けて(Dynabook)}
  \photo{scale=0.28}{Smalltalk-76.png}
        {\tiny 写真: ウィキメディア /  SUMIM.ST / \\
          AltoやNoteTakerで動作した
          アラン・ケイ達の暫定Dynabook環境(Smalltalk-76、同-78の頃) /
          CC-BY-SA 4.0}
  \begin{itemize}
    \item GUI,マウス,パーソナルコンピュータ
  \end{itemize}
  \vfill
\end{frame}

%=========================================================================
\subsection{第4世代(1980〜,PCの時代)}
\begin{frame}
  \frametitle{IBM PC}
  \photo{scale=0.2}
{Bundesarchiv_B_145_Bild-F077948-0006,_Jugend-Computerschule_mit_IBM-PC.jpg}
      {\tiny
          写真: IBM PC \\
          ウィキメディア /
          Bundesarchiv, B 145 Bild-F077948-0006 /
          Engelbert Reineke / CC-BY-SA 3.0 de
      }
  \begin{itemize}
    \item 16bit,32bit,64bit
    \item IBM PC(MS-DOS,Windows)
    \item 互換機
  \end{itemize}
  \vfill
\end{frame}

%=========================================================================
\begin{frame}
  \frametitle{Apple Macintosh}
  \photo{scale=0.2}{Macintosh_128k_transparency.png}
      {\tiny
          写真: 初代Macintosh \\
          ウィキメディア / w:User:Grm wnr / 
          File:Macintosh 128k transparency.png /GFDL
      }
  \begin{itemize}
    \item 16bit,32bit,64bit
    \item Macintosh(Mac OS, Mac OS X, OS X, macOS)
  \end{itemize}
  \vfill
\end{frame}

%=========================================================================
\begin{frame}
  \frametitle{オペレーティングシステムの系統図}
  \fig{scale=0.32}{tree-crop.pdf}
  \vfill
\end{frame}

%=========================================================================
%\section{練習問題}
%\begin{frame}
%  \frametitle{練習問題}
%  \vfill
%  \begin{center}
%    \textbf{\Huge 練習問題}
%  \end{center}
%  \vfill
%\end{frame}

%=========================================================================
\begin{frame}
  \frametitle{練習問題(1)}
  次の言葉の意味を説明しなさい.
  \begin{itemize}
    \item オペレーティングシステム
    \item カーネル
    \item 拡張マシン
    \item 抽象化
    \item 仮想化
    \item 時分割多重
    \item 空間分割多重
    \item プロセス
    \item バッチモニタ
    \item 実行モード
    \item 記憶保護
    \item 仮想記憶
    \item マルチプログラミング
    \item タイムシェアリング
  \end{itemize}
  \vfill
\end{frame}

%=========================================================================
\begin{frame}
  \frametitle{練習問題(2)}
  \begin{itemize}
    \item 抽象化の例をいくつか挙げなさい.
      \vfill
    \item 仮想化の例をいくつか挙げなさい.
      \vfill
    \item 自分が使用しているパソコンのOSは何?
      \vfill
    \item 自分が使用しているスマホのOSは何?
      \vfill
  \end{itemize}
  \vfill
\end{frame}

\end{document}
