%\documentclass[dvipdfmx]{beamer}
\documentclass{beamer}                   % lualatex の場合
\usepackage{mySld}

\begin{document}
\title[OS]{オペレーティングシステム\\第3章 CPUの仮想化}
\date{}

\begin{frame}
  \titlepage
\end{frame}

%\begin{frame}
%  \frametitle
%  \tableofcontents
%\end{frame}

\section{時分割多重}
\begin{frame}
  \frametitle{時分割多重によるCPUの仮想化}
  \fig{scale=0.7}{virtualCPU-crop.pdf}
  \begin{itemize}
    \item 時分割多重:CPUが実行するプロセスを次々切換える。
    \item ディスパッチ:プロセスにCPUを割り付ける。(実行開始)
    \item ディスパッチャ:ディスパッチするプログラムのこと。
  \end{itemize}
\end{frame}

\begin{frame}
  \frametitle{CPUの構造(参考)}
  \fig{scale=0.6}{cpuBlock-crop.pdf}
\end{frame}

\begin{frame}
  \frametitle{プロセスの構造(参考)}
  \fig{scale=0.5}{procOrganization-crop.pdf}
\end{frame}

\section{プロセスの状態}
\begin{frame}
  \frametitle{プロセスの状態遷移}
  \fig{scale=0.6}{procState-crop.pdf}
  \begin{itemize}
    \item 基本的な三つの状態
    \item 六つの状態遷移
  \end{itemize}
\end{frame}

\section{プロセスの切換え}
\begin{frame}
  \frametitle{プロセスの切換え}
  \fig{scale=0.33}{procSwitch-crop.pdf}
\end{frame}

\begin{frame}
  \frametitle{オペレーティングシステムの構造(参考)}
  \fig{scale=0.4}{osOrganization-crop.pdf}
  \begin{itemize}
    \item 割込みハンドラ
    \item サービスモジュール
    \item ディスパッチャ
  \end{itemize}
\end{frame}

\begin{frame}
  \frametitle{プロセスの切換えの例}
  \fig{scale=0.33}{procSwitchInst-crop.pdf}
\end{frame}

\section{PCB(Process Control Block)}
\begin{frame}
  \frametitle{PCBの内容}
\begin{itemize}
\item 仮想CPU
\item プロセス番号
\item 状態(Running,Waiting,Ready等)
\item 優先度
\item 統計情報(CPU利用時間等)
\item 次回のアラーム時刻
\item 親プロセス
\item 子プロセス一覧
\item シグナルハンドリング
\item 使用中のメモリ
\item オープン中のファイル
\item カレントディレクトリ
\item プロセス所有者のユーザ番号
\item PCBのリストを作るためのポインタ
\end{itemize}
\end{frame}

\begin{frame}
  \frametitle{PCBのリスト}
  \fig{scale=0.4}{procQueue-crop.pdf}
\end{frame}

\section{TacOSのCPU仮想化}
\begin{frame}
  \frametitle{TacOSのPCB}
  \begin{itemize}
    \item 仮想CPU(SP)
    \item プロセス番号(pid)
    \item 状態(stat)
    \item 優先度(nice, enaice)
    \item プロセステーブルのインデクス(idx)
    \item イベント用カウンタとセマフォ(evtCnt,evtSem)
    \item プロセスのアドレス空間(memBase,memLen)
    \item プロセスの親子関係の情報(parent,exitStat)
    \item オープン中のファイル一覧(fds[])
    \item PCBリストの管理(prev,next)
    \item スタックオーバーフローの検知(magic)
  \end{itemize}
\end{frame}

\begin{frame}[fragile]
  \frametitle{TaCのPCB(前半)}
  \lstinputlisting
      [language={C},lastline=18,frame=tlr]{Lst/pcb.hmm}
\end{frame}

\begin{frame}[fragile]
  \frametitle{TaCのPCB(後半)}
  \lstinputlisting
      [language={C},firstline=20,firstnumber=20,frame=lrb]
      {Lst/pcb.hmm}
\end{frame}

\begin{frame}
  \frametitle{TacOSのメモリ配置}
  \fig{scale=0.4}{tacosMemMap-crop.pdf}
\end{frame}

\begin{frame}
  \frametitle{TacOSの実行可能列}
  \begin{itemize}
    \item {\tt yield}
    \item {\tt dispatch}
    \item 実行可能列
      \fig{scale=0.5}{tacosReadyQueue-crop.pdf}
  \end{itemize}
\end{frame}

\begin{frame}[fragile]
  \frametitle{プロセス切換えプログラム(yield())}
  \lstinputlisting
      [language=,lastline=23,frame=tlr]{Lst/dispatch.s}
\end{frame}

\begin{frame}[fragile]
  \frametitle{プロセスの切換えプログラム(dispatch())}
  \lstinputlisting
      [language=,firstline=24,firstnumber=24,frame=lrb]
      {Lst/dispatch.s}
\end{frame}

\section{スレッド(Thread)}
\begin{frame}
  \frametitle{マルチプログラミングを用いないWebサーバ}
  \fig{scale=0.6}{singleProcSingleClient-crop.pdf}
\end{frame}

\begin{frame}
  \frametitle{マルチプログラミングを用いないWebサーバ}
  \fig{scale=0.6}{singleProcMultiClient-crop.pdf}
\end{frame}

\begin{frame}
  \frametitle{マルチプログラミングを用いるWebサーバ}
  \fig{scale=0.6}{multiProc-crop.pdf}
\end{frame}

\begin{frame}
  \frametitle{マルチプログラミングを用いるWebサーバ}
  \fig{scale=0.6}{multiThread-crop.pdf}
\end{frame}

\begin{frame}
  \frametitle{ユーザスレッドとカーネルスレッド}
  \fig{scale=0.5}{kernelThread-crop.pdf}
\end{frame}

\begin{frame}
  \frametitle{ユーザスレッドとカーネルスレッド}
  \fig{scale=0.5}{userThread-crop.pdf}
\end{frame}

\begin{frame}
  \frametitle{M個のスレッドで手分けをして合計を計算する様子}
  \fig{scale=0.6}{threadedSum-crop.pdf}
\end{frame}

\begin{frame}[fragile]
  \frametitle{M個のスレッドで合計を計算する(前半)}
  \lstinputlisting
      [language={C},lastline=21,frame=tlr]{Lst/threadTest.c}
\end{frame}

\begin{frame}[fragile]
  \frametitle{M個のスレッドで合計を計算する(後半)}
  \lstinputlisting
      [language={C},firstline=22,firstnumber=22,frame=lrb]
      {Lst/threadTest.c}
\end{frame}

\begin{frame}[fragile]
  \frametitle{M個のスレッドで合計を計算する(後半)}
  \tbl{scale=0.7}{threadTimeTbl.pdf}
  \tbl{scale=0.8}{threadTimeGrph.pdf}
  6コアのMac Proで計測\\
  (Hyper-Threadingのお陰で6コアと12コアの中間的な振舞)
\end{frame}

\end{document}
