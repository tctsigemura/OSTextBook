\RequirePackage{luatex85}
\documentclass{standalone}
\usepackage{luatexja}                              % lualatex の場合
\usepackage[hiragino-pron]{luatexja-preset}        % ヒラギノフォント
\begin{document}
\begin{tabular}{l | l | r | r | r}\hline\hline
  \multicolumn{1}{c|}{パラメータ} &
  \multicolumn{1}{c|}{意味} &
  \multicolumn{1}{c|}{位置} &
  \multicolumn{1}{c|}{長さ} &
  \multicolumn{1}{c}{値の例}\\\hline
  ジャンプ命令     & ジャンプ機械語命令 & 0 & 3 & \texttt{0xeb 0x3e 0x90} \\
  セクタサイズ     & 1セクタのバイト数 & 11 & 2 & 512バイト \\
  クラスタサイズ   & 1クラスタのセクタ数& 13 & 1 & 64セクタ \\
  予約セクタ数     & 予約セクタ数(BPBを含む)& 14 & 2 & 1セクタ \\
  FAT数            & FATを何重に記録するか& 16 & 1 & 2個 \\
  rootDirサイズ    & ディレクトリエントリ数 &  17 & 2 & 512エントリ \\
  総セクタ数16     & ボリュームのサイズ & 19 & 2 & 0 \\
  FATサイズ        & FATのセクタ数    & 22 & 2  & 245セクタ \\
  総セクタ数32     & ボリュームのサイズ & 32 & 4 & 3,999,681セクタ \\
  ボリュームラベル & ボリュームの名前 & 43 & 11 
                   & \texttt{"MICRODRIVE\textvisiblespace"} \\
  ブートプログラム & ブートプログラム & 62 & 448 & \\
  シグネチャ       & フォーマット済みマーク & 510 & 2 & \texttt{0x55 0xaa} \\
  \multicolumn{5}{r}{
    (位置と長さの単位はバイト)}\\
  \multicolumn{5}{r}{
    (値の例はボリュームサイズ2GiB,クラスタサイズ32KiB,FAT16の場合)}\\
  \multicolumn{5}{r}{
    (「総セクタ数16」で表現できない場合は「総セクタ数32」を使用する)}
\end{tabular}
\end{document}
