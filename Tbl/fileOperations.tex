\RequirePackage{luatex85}
\documentclass{standalone}
\usepackage{luatexja}                              % lualatex の場合
\usepackage[hiragino-pron]{luatexja-preset}        % ヒラギノフォント
\usepackage{multirow}
\def\|{\verb|} %|
\begin{document}
\begin{tabular}{l | l}\hline\hline
  \multicolumn{1}{c|}{機能} & \multicolumn{1}{c}{対応するUNIXのAPI}\\\hline
  ファイルを開く     & \|open|システムコール \\
  データを読む       & \|read|システムコール \\
  データを書く       & \|write|システムコール \\
  読み書き位置を移動 & \|lseek|システムコール \\
  ファイルを閉じる   & \|close|システムコール \\
  ファイルの切り詰め & \|truncate|,\|open(... O_TRUNC)|システムコール \\
  ファイルのプログラムを実行 & \|execve|システムコール \\
  ファイルの属性変更 & \|chmod|,\|chown|,\|chgrp|,\|utimes|システムコール \\
  ファイル属性の読出し & \|stat|システムコール \\
\end{tabular}
\end{document}
