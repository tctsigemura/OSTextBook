\begin{thebibliography}{9}

\bibitem{os360}
ウキペディア,
OS/360,
\url{https://ja.wikipedia.org/wiki/OS/360}(2017.10.03 閲覧)

\bibitem{mvs}
ウキペディア,
MVS,
\url{https://ja.wikipedia.org/wiki/Multiple_Virtual_Storage}(2017.10.03 閲覧)

\bibitem{os390}
ウキペディア,
OS/390,
\url{https://ja.wikipedia.org/wiki/OS/390}(2017.10.03 閲覧)

\bibitem{zos}
ウキペディア,
z/OS,
\url{https://ja.wikipedia.org/wiki/Z/OS}(2017.10.03 閲覧)

\bibitem{unix}
ウキペディア,
UNIX(「UNIXおよびUNIX系システムの系統図」を含む),
\url{https://ja.wikipedia.org/wiki/UNIX}(2017.10.03 閲覧)

\bibitem{solaris}
ウキペディア,
Solaris,
\url{https://ja.wikipedia.org/wiki/Solaris}(2017.10.03 閲覧)

\bibitem{aix}
ウキペディア,
AIX,
\url{https://ja.wikipedia.org/wiki/AIX}(2017.10.03 閲覧)

\bibitem{mach}
ウキペディア,
Mach,
\url{https://ja.wikipedia.org/wiki/Mach}(2017.10.03 閲覧)

\bibitem{bsdd}
ウキペディア,
BSDの子孫,
\url{https://ja.wikipedia.org/wiki/BSD%E3%81%AE%E5%AD%90%E5%AD%AB}(2017.10.03 閲覧)

\bibitem{bsd}
ウキペディア,
BSD,
\url{https://ja.wikipedia.org/wiki/BSD}(2017.10.04 閲覧)

\bibitem{386bsd}
ウキペディア,
386BSD,
\url{https://ja.wikipedia.org/wiki/386BSD}(2017.10.04 閲覧)

\bibitem{freebsd}
ウキペディア,
FreeBSD,
\url{https://ja.wikipedia.org/wiki/FreeBSD}(2017.10.03 閲覧)

\bibitem{freenas}
ウキペディア,
FreeNAS,
\url{https://ja.wikipedia.org/wiki/FreeNAS}(2017.10.03 閲覧)

\bibitem{nextstep}
ウキペディア,
NEXTSTEP,
\url{https://ja.wikipedia.org/wiki/NEXTSTEP}(2017.10.03 閲覧)

\bibitem{classicmacos}
ウキペディア,
Classic Mac OS,
\url{https://ja.wikipedia.org/wiki/Classic_Mac_OS}(2017.10.03 閲覧)

\bibitem{dynabook}
ウキペディア,
ダイナブック,
\url{
https://ja.wikipedia.org/wiki/ダイナブック
}(2017.10.03 閲覧)

\bibitem{macos}
ウキペディア,
macOS,
\url{
https://ja.wikipedia.org/wiki/MacOS
}(2017.10.03 閲覧)

\bibitem{ios}
ウキペディア,
iOS (アップル),
\url{
https://ja.wikipedia.org/wiki/IOS_(アップル)
}(2017.10.03 閲覧)

\bibitem{linux}
ウキペディア,
Linux,
\url{
https://ja.wikipedia.org/wiki/Linux
}(2017.10.03 閲覧)

\bibitem{android}
ウキペディア,
Andriod,
\url{
https://ja.wikipedia.org/wiki/Android
}(2017.10.03 閲覧)

\bibitem{msdos}
ウキペディア,
MS-DOS,
\url{
https://ja.wikipedia.org/wiki/MS-DOS
}(2017.10.03 閲覧)

\bibitem{windows}
ウキペディア,
Microsoft Windows(「Windows ファミリー系統図」含む),
\url{
https://ja.wikipedia.org/wiki/Microsoft_Windows
}(2017.10.03 閲覧)

\bibitem{ibmpc81}
ウキペディア,IBM PC,
\url{
https://ja.wikipedia.org/wiki/IBM_PC
}(2017.10.04 閲覧)

\bibitem{svr4}
ウキペディア,UNIX System V,
\url{
https://ja.wikipedia.org/wiki/UNIX_System_V
}(2017.10.04 閲覧)

\bibitem{iebsd}
重村哲至,情報電子工学科電算機室におけるPC-UNIXの歴史,
\url{
http://www2.tokuyama.ac.jp/giga/Sigemura/Public/IeNet/history.html
}(2017.10.03 閲覧)

\bibitem{linux1}
Linux kernel release 1.0,
\url{
https://www.kernel.org/pub/linux/kernel/v1.0/linux-1.0.tar.gz
}(2017.10.04)

\bibitem{third}
Andrew S. Tanenbaum, Herbert Bos:
``The Third Generaton(1965--1980):ICs and Multiproguramming'',
Modern Operating Systems(4th Edition),pp.9-14,
Pearson Education,Inc(2014).

\bibitem{fourth}
Andrew S. Tanenbaum, Herbert Bos:
``The Fourth Generation(1980--Present):Personal Computers'',
Modern Operating Systems(4th Edition),pp.15--19,
Pearson Education,Inc(2014).

\bibitem{key72}
Alan C. Kay:
``A Personal Computer for Children of All Ages'',
Proceeding ACM '72 Proceedings of the ACM annual conference - Volume 1
Article No 1 (1972).

\bibitem{key72J}
アラン・ケイ:すべての年齢の「子供たち」のためのパーソナルコンピュータ,
阿部和広,小学生からはじめるわくわくプログラミング,pp.130--141,
日経BP社(2013).

\bibitem{dynabook2}
アラン・ケイ:Dynabookとは何か?
「すべての年齢の「子供たち」のためのパーソナルコンピュータ」の後日談,
阿部和広,小学生からはじめるわくわくプログラミング,pp.142--149,
日経BP社(2013).

\bibitem{kei}
師尾 潤 他:スーパーコンピュータ「京」のオペレーティングシステム,
\url{http://img.jp.fujitsu.com/downloads/jp/jmag/vol63-3/paper07.pdf}
(2017.10.03 閲覧),
富士通(2012).

\bibitem{zfs}
Marshall Kirk McKusick,
George V. Neville-Neil,
Robert N. M. Watson:The Zettabyte Filesystem,
The Design and Implementation of the FreeBSD Operating System
Second Edition,Pearson Education,Inc(2015).

%\bibitem{cpmJ}
%アンドリュー・S・タネンバウム,
%モダンオペレーティングシステム原著第2版,13ページ,
%ピアソン・エデュケーション(2004).

\bibitem{lines}
Andrew S. Tanenbaum, Herbert Bos:
``INTRODUCTION'',
Modern Operating Systems(4th Edition),pp.1-3,
Pearson Education,Inc(2014).

\bibitem{virtualization}
Andrew S. Tanenbaum, Herbert Bos:
``VIRTUALIZATION AND THE CLOUD'',
Modern Operating Systems(4th Edition),pp.471-516,
Pearson Education,Inc(2014).

\bibitem{vsphere}
ヴイエムウェア株式会社:
``VMware 徹底入門 第3版'',
廣済堂(2012).

\bibitem{ubuntu}
仮想ハードディスクイメージのダウンロード,
\url{https://www.ubuntulinux.jp/download/ja-remix-vhd}
(2017.10.19 閲覧),
Ubuntu Japanese Team(2012).

\bibitem{lightWeight}
Andrew S. Tanenbaum, Herbert Bos:
``Thread Usage'',
Modern Operating Systems(4th Edition),pp.97-102,
Pearson Education,Inc(2014).

\bibitem{hyperThreading}
ウキペディア,ハイパースレッディング・テクノロジー,
\url{
https://ja.wikipedia.org/wiki/%E3%83%8F%E3%82%A4%E3%83%91%E3%83%BC%E3%82%B9%E3%83%AC%E3%83%83%E3%83%87%E3%82%A3%E3%83%B3%E3%82%B0%E3%83%BB%E3%83%86%E3%82%AF%E3%83%8E%E3%83%AD%E3%82%B8%E3%83%BC
}(2017.11.02 閲覧)

\bibitem{AbstractDataType}
B.H.Liskov, S.N.Zilles:``Programming with Abstract Data Type'',
SIGPLAN Notices,9,4,pp.50-59(1974).

\bibitem{ia32Segmentation}
John H. Crawford, Patrick P. Gelsinger:``ベースとリミット'',
80386プログラミング, 工学社, pp.413-414(1988).

\bibitem{ia32SegmentReg}
John H. Crawford, Patrick P. Gelsinger:
``セグメント部:セグメント・レジスタ'',
80386プログラミング, 工学社, pp.48-50(1988).

\bibitem{ia32SegmentHiddenReg}
John H. Crawford, Patrick P. Gelsinger:
``デスクリプタ用の裏レジスタ'',
80386プログラミング, 工学社, pp.420-421(1988).

\bibitem{invertedPageTable}
Albert Chang, Mark F. Mergen:
``801 storage: architecture and programming'',
ACM Transactions on Computer Systems, 6, 1,
pp.28-50 (1988).
\end{thebibliography}
