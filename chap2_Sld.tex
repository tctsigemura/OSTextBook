\documentclass[dvipdfmx]{beamer}

% リスト環境
\usepackage{listings,jlisting}
%\usepackage{listings}
\def\lstlistingname{リスト}
\lstset{language=,
  numbers=left,
%  numbers=none,
  basicstyle={\scriptsize\ttfamily},
  columns=[l]{fullflexible},
  keepspaces=true,
%  frame=shadowbox,
  frame=single,
  commentstyle=\slshape
}

\newcommand{\fig}[2]{\begin{center}\includegraphics[#1]{Fig/#2}\end{center}}
\newcommand{\tbl}[2]{\begin{center}\includegraphics[#1]{Tbl/#2}\end{center}}

\begin{document}
\title[OS]{オペレーティングシステム\\第2章 前提知識}
\date{}

\begin{frame}
  \titlepage
\end{frame}

\begin{frame}
  \frametitle
  \tableofcontents
\end{frame}

\section{コンピュータのハードウェア構成}
\begin{frame}
  \frametitle{ハードウェア構成}
  \fig{scale=0.41}{hardBlock-crop.pdf}
  \begin{itemize}
    \item CPU,メモリ,タイマー,グラフィックアダプタ
    \item SATAホストコントローラ,USBコントローラ
    \item ネットワークアダプタ,バス
  \end{itemize}
\end{frame}

\section{CPUの構成}
\begin{frame}
  \frametitle{CPUの構成}
  \fig{scale=0.5}{cpuBlock-crop.pdf}
  \begin{itemize}
    \item PSW(Program Status Word)
    \item CPUレジスタ
  \end{itemize}
\end{frame}

\section{最近のコンピュータの実際の構成}
\begin{frame}
  \frametitle{デスクトップ・パーソナルコンピュータ}
  \fig{scale=0.5}{intelDesktop-crop.pdf}
  \begin{itemize}
    \item CPU
    \item コア(Core)
  \end{itemize}
\end{frame}

\begin{frame}
  \frametitle{サーバコンピュータ}
  \fig{scale=0.4}{intelServer-crop.pdf}
\end{frame}

\section{オペレーティングシステムの構造}
\begin{frame}
  \frametitle{オペレーティングシステムの構造}
  \fig{scale=0.49}{osOrganization-crop.pdf}
\end{frame}

\begin{frame}
  \frametitle{プロセスの構造}
  \fig{scale=0.5}{procOrganization-crop.pdf}
\end{frame}

\section{カーネルの構成方式}
\begin{frame}
  \frametitle{単層カーネル(モノリシック・カーネル)}
  \fig{scale=0.49}{osOrganization-crop.pdf}
\end{frame}

\begin{frame}
  \frametitle{マイクロカーネル(micro-kernel)}
  \fig{scale=0.5}{microkernel-crop.pdf}
\end{frame}

\section{もう一つの仮想マシン}
\begin{frame}
  \frametitle{Type 2 ハイパーバイザ}
  \fig{scale=0.6}{type2Hypervisor-crop.pdf}
\end{frame}

\begin{frame}
  \frametitle{Type 1 ハイパーバイザ}
  \fig{scale=0.6}{type1Hypervisor-crop.pdf}
\end{frame}


\end{document}
